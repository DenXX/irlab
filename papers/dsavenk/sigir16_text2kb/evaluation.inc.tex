
The experiments in this work were conducted on WebQuestions dataset, presented in \cite{Berant:EMNLP13}.
Recently, it gained a lot of attention and became the benchmark dataset for KBQA.
The performance of existing systems and our Text2KB is presented in Table \ref{table:webquestions_results}.

\begin{table}
\caption{Performance of the Text2KB system on WebQuestions dataset}
\label{table:webquestions_results}
\begin{tabular}{| p{2.7cm} | c | c | c | c | }
\hline
System & avg Re & avg Pr & F1 of avg & avg F1 \\
\hline
SemPre \cite{Berant:EMNLP13} & 0.413 & 0.480 & 0.444 & 0.357\\
Subgraph Embeddings \cite{BordesCW14:emnlp} & - & - & 0.432 & 0.392\\
ParaSemPre \cite{berant2014semantic} & 0.466 & 0.405 & 0.433 & 0.399\\
Jacana \cite{yao2014information} & 0.458 & 0.517 & 0.486 & 0.330\\
Kitt AI \cite{yao-scratch-qa-naacl2015} & 0.545 & 0.526 & 0.535 & 0.443\\
AgendaIL \cite{berant2015imitation} & 0.557 & 0.505 & 0.530 & 0.497\\
STAGG \cite{yih2015semantic} & 0.607 & 0.528 & 0.565 & 0.525\\
\hline
Our baseline \cite{ACCU:2015} & 0.604 & 0.498 & 0.546 & 0.494\\
Text-only baseline & & & & \\
Text2KB & & & & \\
\hline
\end{tabular}
\end{table}


\subsection{Ablation Study}

To study the effect of different components we made an ablation study, and the results are presented in Table \ref{table:webquestions_ablation}.
The following notations are used to represent different components of our system: T - notable type model, DF - using date range filter, TF - using notable type based filter, E - using web search results to detect question entities, W - using web search snippets and documents features, CQA - using cqa term-relation scores to generate features, CW - using entity pair language model from ClueWeb for feature generation.

\begin{table}
\caption{Ablation study}
\label{table:webquestions_ablation}
\begin{tabular}{| p{4cm} | c | c | c | }
\hline
System & avg Re & avg Pr &  avg F1 \\
\hline
baseline & 0.604 & 0.498 & 0.494\\
baseline +E & & & \\
baseline +E+T & & & \\
baseline +E+T+W & & & \\
baseline +E+T+CQA & & & \\
baseline +E+T+CW & & & \\
\hline
Text2KB -E & & & \\
Text2KB -W & & & \\
Text2KB -CQA & & & \\
Text2KB -CL & & & \\
\hline
\end{tabular}
\end{table}

As we can see ...