% This is "sig-alternate.tex" V2.1 April 2013
% This file should be compiled with V2.5 of "sig-alternate.cls" May 2012
%
% This example file demonstrates the use of the 'sig-alternate.cls'
% V2.5 LaTeX2e document class file. It is for those submitting
% articles to ACM Conference Proceedings WHO DO NOT WISH TO
% STRICTLY ADHERE TO THE SIGS (PUBS-BOARD-ENDORSED) STYLE.
% The 'sig-alternate.cls' file will produce a similar-looking,
% albeit, 'tighter' paper resulting in, invariably, fewer pages.
%
% ----------------------------------------------------------------------------------------------------------------
% This .tex file (and associated .cls V2.5) produces:
%       1) The Permission Statement
%       2) The Conference (location) Info information
%       3) The Copyright Line with ACM data
%       4) NO page numbers
%
% as against the acm_proc_article-sp.cls file which
% DOES NOT produce 1) thru' 3) above.
%
% Using 'sig-alternate.cls' you have control, however, from within
% the source .tex file, over both the CopyrightYear
% (defaulted to 200X) and the ACM Copyright Data
% (defaulted to X-XXXXX-XX-X/XX/XX).
% e.g.
% \CopyrightYear{2007} will cause 2007 to appear in the copyright line.
% \crdata{0-12345-67-8/90/12} will cause 0-12345-67-8/90/12 to appear in the copyright line.
%
% ---------------------------------------------------------------------------------------------------------------
% This .tex source is an example which *does* use
% the .bib file (from which the .bbl file % is produced).
% REMEMBER HOWEVER: After having produced the .bbl file,
% and prior to final submission, you *NEED* to 'insert'
% your .bbl file into your source .tex file so as to provide
% ONE 'self-contained' source file.
%
% ================= IF YOU HAVE QUESTIONS =======================
% Questions regarding the SIGS styles, SIGS policies and
% procedures, Conferences etc. should be sent to
% Adrienne Griscti (griscti@acm.org)
%
% Technical questions _only_ to
% Gerald Murray (murray@hq.acm.org)
% ===============================================================
%
% For tracking purposes - this is V2.0 - May 2012

\documentclass{sig-alternate-05-2015}

\newcommand{\eg}[0]{{\em e.g. }}
\newcommand{\etc}[0]{{\em etc. }}
\newcommand{\ie}[0]{{\em i.e. }}

\begin{document}

% Copyright
\setcopyright{acmcopyright}
%\setcopyright{acmlicensed}
%\setcopyright{rightsretained}
%\setcopyright{usgov}
%\setcopyright{usgovmixed}
%\setcopyright{cagov}
%\setcopyright{cagovmixed}


% DOI
\doi{10.475/123_4}

% ISBN
\isbn{123-4567-24-567/08/06}

%Conference
\conferenceinfo{SIGIR '16}{June 18--20, 2016, Pisa, Italy}

\acmPrice{\$15.00}

%
% --- Author Metadata here ---
\conferenceinfo{SIGIR}{'16 Pisa, Italy}
%\CopyrightYear{2007} % Allows default copyright year (20XX) to be over-ridden - IF NEED BE.
%\crdata{0-12345-67-8/90/01}  % Allows default copyright data (0-89791-88-6/97/05) to be over-ridden - IF NEED BE.
% --- End of Author Metadata ---

\title{Text2KB: Using Text Document Collections for Knowledge Base Question Answering}

%
% You need the command \numberofauthors to handle the 'placement
% and alignment' of the authors beneath the title.
%
% For aesthetic reasons, we recommend 'three authors at a time'
% i.e. three 'name/affiliation blocks' be placed beneath the title.
%
% NOTE: You are NOT restricted in how many 'rows' of
% "name/affiliations" may appear. We just ask that you restrict
% the number of 'columns' to three.
%
% Because of the available 'opening page real-estate'
% we ask you to refrain from putting more than six authors
% (two rows with three columns) beneath the article title.
% More than six makes the first-page appear very cluttered indeed.
%
% Use the \alignauthor commands to handle the names
% and affiliations for an 'aesthetic maximum' of six authors.
% Add names, affiliations, addresses for
% the seventh etc. author(s) as the argument for the
% \additionalauthors command.
% These 'additional authors' will be output/set for you
% without further effort on your part as the last section in
% the body of your article BEFORE References or any Appendices.

\numberofauthors{2} %  in this sample file, there are a *total*
% of EIGHT authors. SIX appear on the 'first-page' (for formatting
% reasons) and the remaining two appear in the \additionalauthors section.
%
\author{
% 1st. author
\alignauthor
Denis Savenkov
       \email{dsavenk@emory.edu}
% 2nd. author
\alignauthor
Eugene Agichtein\\
       \email{eugene@mathcs.emory.edu}
}
% There's nothing stopping you putting the seventh, eighth, etc.
% author on the opening page (as the 'third row') but we ask,
% for aesthetic reasons that you place these 'additional authors'
% in the \additional authors block, viz.
%\additionalauthors{Additional authors: John Smith (The Th{\o}rv{\"a}ld Group,
%email: {\texttt{jsmith@affiliation.org}}) and Julius P.~Kumquat
%(The Kumquat Consortium, email: {\texttt{jpkumquat@consortium.net}}).}
%\date{30 July 1999}
% Just remember to make sure that the TOTAL number of authors
% is the number that will appear on the first page PLUS the
% number that will appear in the \additionalauthors section.

\maketitle
\begin{abstract}

One of the major challenges for knowledge base question answering systems (KBQA) is to map phrases from natural language questions to knowledge base (KB) entities and predicates.
Existing models are typically based on a lexicon built from a limited data during training and ignore the vast amount of text data, effectively used by traditional document-based question answering systems.
In this work we propose to use entity linking to connect structured KB with unstructured text data, which opens up an opportunity to combine QA approaches.
Our system utilizes various different text resources for candidate generation as well as ranking and achieves a state-of-the-art performance of \textbf{XX.XX} average F1 score on WebQuestions knowledge base question answering dataset.



\end{abstract}


% The code below should be generated by the tool at
% http://dl.acm.org/ccs.cfm
% Please copy and paste the code instead of the example below. 
%
%\begin{CCSXML}
%<ccs2012>
% <concept>
%  <concept_id>10010520.10010553.10010562</concept_id>
%  <concept_desc>Computer systems organization~Embedded systems</concept_desc>
%  <concept_significance>500</concept_significance>
% </concept>
% <concept>
%  <concept_id>10010520.10010575.10010755</concept_id>
%  <concept_desc>Computer systems organization~Redundancy</concept_desc>
%  <concept_significance>300</concept_significance>
% </concept>
% <concept>
%  <concept_id>10010520.10010553.10010554</concept_id>
%  <concept_desc>Computer systems organization~Robotics</concept_desc>
%  <concept_significance>100</concept_significance>
% </concept>
% <concept>
%  <concept_id>10003033.10003083.10003095</concept_id>
%  <concept_desc>Networks~Network reliability</concept_desc>
%  <concept_significance>100</concept_significance>
% </concept>
% </ccs2012>  
%\end{CCSXML}

%\ccsdesc[500]{Computer systems organization~Embedded systems}
%\ccsdesc[300]{Computer systems organization~Redundancy}
%\ccsdesc{Computer systems organization~Robotics}
%\ccsdesc[100]{Networks~Network reliability}
%
% End generated code
%
%
%  Use this command to print the description
%
%\printccsdesc

% We no longer use \terms command
%\terms{Theory}

% \keywords{ACM proceedings; \LaTeX; text tagging}

\section{Introduction}
It has long been recognized that searchers prefer concise and specific answers, rather than lists of document results. In particular, factual, or factoid questions, have been an active focus of research for decades due to both practical importance and relatively objective evaluation criteria. As a particularly important example, a large proportion of Web search queries are looking for entities or their attributes (\textbf{CITATION}), a setting on which we focus in this work. 

Two relatively separate approaches Question Answering (QA) have emerged: text-centric, or Text-QA and knowledge-base-centric, or KB-QA. In the more traditional, text-QA approach, QA systems used text document collections to retrieve passages relevant to a question and to extract candidate answers \cite{Vrandecic:2014:WFC:2661061.2629489}. Unfortunately, an unstructured text passage does not provide explicit information about the candidate entities, and has to be inferred from the question text. The KB-QA approach, which evolved from the database community, relies on large scale knowledge bases, such as dbPedia \cite{auer2007dbpedia}, Freebase \cite{Bollacker:2008:FCC:1376616.1376746} and WikiData \cite{Vrandecic:2014:WFC:2661061.2629489}, which store a vast amount of general knowledge about different kinds of entities.
This information, encoded as \texttt{[subject, predicate, object]} RDF triples, can be effectively queried using structured query languages, such as SPARQL.

Both approaches need to eventually deal with natural language questions, in which information needs are expressed by the vast majority of users. While question understanding is difficult in itself, this setting is particularly challenging for KB-QA systems, as it requires a translation of a text question into a structured query language. That is challenging for a number of reasons, including the complexity of a KB schema, and many differences between natural language and knowledge representations.  For example, Figure \ref{fig:example_sparql} gives a SPARQL query that retrieves the answer to a relatively simple question \textit{``who was the president of the Dominican Republic in 2010?''} from Freebase.
% The same information can be asked in many different ways, for example: \textit{``who is the dominican republic president in 2010?''}, or \textit{``who was the leader of the dominican republic in 2010?''} \etc


\begin{figure*}
\centering
\begin{lstlisting}[frame=single]
PREFIX : <http://rdf.freebase.com/ns/>
SELECT DISTINCT ?name {
   :m.027rn :government.governmental_jurisdiction.governing_officials ?gov_position .
   ?gov_position :government.government_position_held.basic_title :m.060c4 .
   ?gov_position :government.government_position_held.office_holder ?president .
   ?gov_position :government.government_position_held.from ?from_date .
   ?gov_position :government.government_position_held.to ?to_date .
   FILTER (
       xsd:date(?from_date) <= "2010"^^xsd:date AND
       xsd:date(?to_date) >= "2010"^^xsd:date
   )
   ?president :type.object.name ?name
}
\end{lstlisting}
\caption{SPARQL query that retrieves the answer to the query \textit{``who is the current president of the dominican republic in 2010?''}}
\label{fig:example_sparql}
\end{figure*}

\begin{figure}
\centering
\includegraphics[width=0.45\textwidth]{img/web_search_entitylink}
\caption{Search results for the question ``what year did tut became king?''}
\label{fig:web_search_entitylink}
\end{figure}


Any KB-QA systems must address three challenges, namely question entity identification to anchor the query process; candidate answer entity identification; and candidate ranking. We will show that these challenges can be alleviated by the appropriate use of external textual data. 

The first problem that a KBQA system faces is question entity identification.
The performance of the whole system greatly depends on this stage \cite{yao-scratch-qa-naacl2015}, because it seeds the answer candidate search process.
Question text is often quite short, may contain typos and other problems, that complicate question entity identification.
Existing approaches are usually based on dictionaries that contain entity names, aliases and some other phrases, which were used to refer to entities \cite{SPITKOVSKY12.266}.
These dictionaries are often noisy and incomplete, \eg to answer the question \textit{``what year did tut became king?''} a system needs to detect a mention \textit{``tut''}, which refers to the entity \textit{``Tutankhamun''}.
A mapping \textit{tut $\rightarrow$ ``Tutankhamun''} is missing in the dictionary used by one of the state of the art systems and therefore it couldn't answer this question correctly.
Instead of increasing the dictionary size we propose to use web search results to find variations of question entity names, which can be easier to link to a KB.
This idea has been shown effective in entity linking in web search queries \cite{SMAPH_ERD:2014}, one of the tracks on the Entity Recognition and Disambiguation Challenge 2014\footnote{http://web-ngram.research.microsoft.com/ERD2014/}.
Figure \ref{fig:web_search_entitylink} presents web search results for the query \textit{``what year did tut became king?''}, which shows that indeed many documents mention the full name of the entity, which in turn can be more easily mapped to a KB entity.

After question entities have been identified, the second challenge is exploring their neighborhoods in the KB to generate candidate answers.
A query addresses one or multiple KB predicates, which should be somehow related to words and phrases in the question, and somehow ordered in order to select the best answer.
Existing knowledge base question answering approaches \cite{ACCU:2015,Berant:EMNLP13,berant2014semantic,berant2015imitation,BordesCW14:emnlp,yao2014freebase} rely on some kind of a lexicon, which is learned from manually labeled training data, and supported by  additional resources, such as question paraphrases \cite{berant2014semantic} and weakly labeled sentences from a large text collection \cite{yao2014information}.
However, since manually labeled training data tends to be limited, such lexicons do not cover thousands of different predicates often present in a KB.
By our estimate, in a popular WebQuestions KBQA dataset, the answers to $\sim$5.5\% of test questions (112 out of 2032) involve a predicate that {\em does not appear} in the training set.
For example, an RDF triple \texttt{[Bigos, food.dish.type\_of\_dish1, Stew]} answers a test question \textit{``what are bigos?''}, but there are no questions from the training set that are answered using the same predicate.
In addition, even if the training set contained an example targeting a particular KB predicate, the lexicon might not cover all the other possible ways the same information can be asked about.
For example, a test question in the WebQuestions dataset is \textit{``who is the woman that john edwards had an affair with?''}. This question is similar to, and is answered with a similar query as a training set question \textit{``who did jon gosselin cheat with?''}, but the word \textit{affair} isn't used in the training set.
On the other hand, traditional text-based question answering systems benefit from the redundancy of information on the Web, where the same information is stated in many different ways in many documents \cite{Lin:2007:EPU:1229179.1229180}.
This increases the chances of a good lexical match between a question and answer statements, which makes even some relatively simple counting-based techniques quite effective \cite{brill2002analysis}.
Thus, to address this challenge, our work adapts ideas from text-based question answering to enrich the representation of candidate structured queries with additional text documents and fragments, that can help to select the best answer.
For example, the right part of the Figure \ref{fig:model} shows web search results, a community question answering page, and text fragments mentioning pairs of entities, that can be useful to answer the question about John Edwards' affair.
Finally, the third challenge of a KBQA system is how to select the best candidate answer among many candidates. We show that enriching the candidate answers with features derived from external text results, significantly improves the ranking. 

To summarize, our contributions are three-fold:
\begin{itemize}
\item A novel ``hybrid'' knowledge base question answering system, which uses both structured data from a knowledge base and unstructured natural language resources connected via entity links. Section \ref{section:method} describes the architecture of our system, and Section \ref{section:eval} shows that this fusion improves the performance of a state of the art KBQA system.
\item Novel data sources and techniques for knowledge base question answering, via entity linking. We introduce three techniques: enhancing question entity identification by analyzing web search results (Section \ref{section:method:web}); improving predicate matching by mining CQA data (Section \ref{section:method:cqa}); and improving candidate ranking by incorporating text-corpus statistics (Section \ref{section:method:clueweb}).
\item Comprehensive empirical analysis of our system on a popular WebQuestions benchmark, demonstrating that using additional text resources can improve the performance of a state-of-the-art KBQA system (Section \ref{section:eval}). In addition, we conduct an extensive analysis of the system to identify promising directions for future improvements (Section \ref{section:analysis}).
\end{itemize}

Taken together, this work introduces a number of techniques of using external text that significantly improve the performance of the KBQA approach. More broadly, our work bridges the gap between Text-QA and KB-QA worlds, demonstrating an important step forward towards combining unstructured and structured data for question answering. 



% -------------------------------------------

%There are many problems in KBQA:
%\begin{itemize}
%\item lexical variations, we can call the same thing in million ways
%\item representation variation - similar data can be represented in multiple ways, e.g. capital of the state in Australia location.australian\_state.capital\_city, while in the US you will have totally different predicate. HOWEVER, these are old predicate and marked deprecated. There is another predicate that should be the same for both cases.
%\item Incomplete, some data is simply missing or details are not present. E.g. who is the woman that john edwards had an affair with?, there is a triple that says that he had sexual relationships with Rielle Hunter, but there is no details...
%\item Related to the previous - many predicates are simply not present. There is something related, but not exactly what is asked about. Example: where did andy murray started playing tennis? We can find the answer entity, but the triple won't say that he started playing there.
%\end{itemize}

%In \cite{Sun:2015:ODQ:2736277.2741651} authors report pretty low score for Sempre on TREC and Bing QA datasets.

% Questions and corresponding answers are often expressed differently and researchers in question answering studied different ways to bridge this lexical gap, \eg using translation models \cite{Murdock:2005:TMS:1220575.1220661} and distributional semantics \cite{yu2014deep}.




\section{Methodology}
We now introduce our system, called Text2KB, that expands upon the basic KBQA model by incorporating external textual sources throughout the QA process. The general architecture and an example use case of Text2KB is presented on Figure \ref{fig:model}. 
The left part of the figure roughly corresponds to the architecture of existing information extraction approaches to KBQA.
The right part introduces additional external text data sources, specifically
we investigate the use of web search results, community question answering (CQA) data, and a collection of documents with detected KB entity mentions.
We demonstrate how these data sources can help with the main challenges in KBQA, \ie question topical entity identification, predicate scoring and answer candidates ranking.
% Recall that the main challenges in KBQA are linking topical entities in the question to the KB; identifying candidate answers in the neighborhood around the question entities; and ranking the candidates. In the rest of this section we present our approach to solving each of these challenges, by: using web search results, CQA data, and external corpus statistics. 


\subsection{Web search results for KBQA}
\label{section:method:web}

Traditional Text-QA systems rely on search results to retrieve relevant documents, which are then used to extract answers to users' questions.
Relevant search results mention question entities multiple times and in various forms, which can be helpful for question topical entity identification \cite{SMAPH_ERD:2014}.
Furthermore, retrieved document set often contains multiple statements of the answer, which can be a strong signal for candidate ranking \cite{Lin:2007:EPU:1229179.1229180}.

To obtain related web search results, Text2KB issues the question as a query to a commercial web search engine\footnote{https://datamarket.azure.com/dataset/bing/search}, extracts top 10 result snippets and the corresponding documents.
Next, it detects KB entity mentions in both snippets and documents using the same method it applies to the question itself.

\textbf{Question entity identification}.
Question text provides only a limited context for entity disambiguation and linking; additionally, the entity name can be misspelled or an uncommon variation used.
This complicates the task of entity identification, which is the foundation of whole question answering process.
Fortunately, web search results help with these problems, as they usually contain multiple mentions of the same entities and provide more context for disambiguation.
Text2KB uses the search result snippets to \textit{expand} the set of detected question entities.
More specifically, we count the frequencies of each entity mentioned in search snippets, and most popular ones with names similar to some of the question terms are added to the list of topical entities.
The goal of this similarity condition is to keep only entities that are likely mentioned in the question text and filter out simply related entities.
To estimate the similarity between a name and question tokens we use Jaro-Winkler string distance, an entity is added to the list of question entities if at least one of its tokens $e_t$ has high similarity with one of the question tokens $q_t$ excluding stopwords ($Stop$):
$$\max_{e_t \in M\backslash Stop, q_t \in Q\backslash Stop} 1 - dist(e_t, q_t) \geq 0.8$$

\textbf{Answer candidate features}.
The information stored in KBs can also be present in other formats, \eg text statements.
For example, on Figure \ref{fig:web_search_entitylink} multiple snippets mention the date when Tutankhamun became the king.
Text-QA systems use such passages to extract answer to users' questions.
However, text provides very little context information about the mentioned entities, and systems have to infer the useful details, \eg entity types, which can be problematic \cite{yih2015semantic}.
On the other hand, KBQA systems can utilize all the available KB knowledge about the entities in a candidate answer, and would benefit from additional text-based information to improve ranking.
More specifically, we perform the following:

\begin{enumerate}
\setlength\itemsep{0em}
\item Precompute term and entity IDFs. We used Google n-grams corpus to approximate terms IDF by collection frequencies and available ClueWeb Freebase entity annotations\footnote{http://lemurproject.org/clueweb09/FACC1/} to compute entity IDFs
\item Each snippet and document is represented by two TF-IDF vectors of lowercased tokens and mentioned entities
\item In addition, vectors of all snippets and all documents are merged together to form combined token and entity vectors
\item Each answer candidate is also represented as TF-IDF vectors of terms (from entity names) and entities
\item We compute cosine similarities between answer and each snippet and document token and entity vectors. This gives us 10 similarity scores for every document for token vectors and 10 similarities for entity vectors. We take average and maximum scores as features.
\item We do the same for the combined document and use cosine similarities as features.
\end{enumerate}

\subsection{CQA data for Matching Questions to Predicates}
\label{section:method:cqa}

Recall that a major challenge in KBQA is that natural language questions do not easily or uniquely map to entities and predicates in a KB.
An established approach for this task is supervised machine learning, which requires labeled examples of questions and the corresponding answer to learn this mapping, which can be expensive.
% On the other hand, there are huge archives of questions and answers posted by real users on various community question answering websites, \eg Figure \ref{fig:cqa_example}.
% Unfortunately, manual labeling of questions with answers is expensive, and necessarily contains only a small fraction of the different ways the same KB predicate can be inquired about using natural language questions.
Researchers have proposed to use weakly supervised methods to extend a lexicon with mappings learned from \textit{single sentence statements} mentioning entity pairs in a large corpus \cite{yao2014information}.
However, the language used in questions to query about a certain predicate may differ from the language used in statements.
A recent work~\cite{savenkov-EtAl:2015:SRW} demonstrated how distant supervision assumption can be applied to question-answer pairs from CQA archives for a related task of information extraction for knowledge base completion.
In a similar way, we use weakly labeled collection of question-answer pairs to compute associations between question terms and predicates to \textit{extend} system's lexicon (Figure \ref{fig:cqa_example}).
We should emphasize, that this data doesn't replace the mappings, learned from single sentence statements, which are already used by our baseline systems, but rather extend it.
Weakly labeled question-answer pairs are usually more noisy than sentences, but can provide complementary information \cite{savenkov-EtAl:2015:SRW}.

\begin{figure}
\centering
\fbox{
\includegraphics[width=0.4\textwidth]{img/cqa_example}
}
\caption{Example of a question and answer pair from Yahoo! Answers CQA website}
\label{fig:cqa_example}
\end{figure}

For our experiments we use 4.4M questions from Yahoo! WebScore L6 dataset\footnote{https://webscope.sandbox.yahoo.com/catalog.php?datatype=l}.
Question and answer texts were run through an entity linker, that detected mentions of Freebase entities.
Next, we use distant supervision assumption to label each question-answer pair with predicates between entities mentioned in the question and in the answer.
% As a result, we have a set of questions, annotated with KB predicates, which are, often incorrectly, assumed to answer the question.
This labels are used to learn associations between question terms and predicates by computing pointwise mutual information scores (PMI) for each term-predicate pair.
Examples of scores for some terms are given in Table \ref{table:cqa_npmi}.

\begin{table}
\caption{Examples of term-predicate pairs with high PMI scores, computed using distant supervision from a CQA collection}
\label{table:cqa_npmi}
\begin{tabular}{| p{1cm} | p{5.5cm} | p{0.75cm} |}
\hline
Term & Predicate & PMI score\\
\hline
born & people.person.date\_of\_birth & 3.67\\
 & people.person.date\_of\_death & 2.73\\
 & location.location.people\_born\_here & 1.60\\
\hline
kill & people.deceased\_person.cause\_of\_death & 1.70\\
& book.book.characters & 1.55\\
\hline
currency & location.country.currency\_formerly\_used & 5.55 \\
& location.country.currency\_used & 3.54 \\
\hline
school & education.school.school\_district & 4.14 \\
& people.education.institution & 1.70\\
& sports.school\_sports\_team.school & 1.69 \\
%\hline
%illness & medicine.symptom.symptom\_of & 2.11\\
%& medicine.decease.causes & 1.68\\
%& medicine.disease.treatments & 1.59\\
\hline
win & sports.sports\_team.championships & 4.11\\
& sports.sports\_league.championship & 3.79\\
\hline
\end{tabular}
\end{table}

In Text2KB we take candidate answer predicates and look up the PMI scores between them and terms in the question (missing pairs are given a score of 0).
From this list of score we compute minimum, average and maximum and add these values to the feature list.
Since this kind of data is usually sparse, we also use pretrained word2vec word embeddings\footnote{https://code.google.com/p/word2vec/}.
We compute predicate embeddings by taking a weighted average of term vectors from predicate's PMI table.
Each term vector is weighted by its PMI value (terms with negative score are skipped).
Then, we compute cosine similarities between predicate vector and each of the question term vectors and take their minimum, average, maximum as features.
Finally, we average embeddings of question terms and compute its cosine similarity with the predicate vector.

\subsection{Estimating Entity Associations}
\label{section:method:clueweb}

A key step for ranking candidate answers is to estimate whether the question and answer entities are related in a way asked in the question.
Existing KBQA approaches usually focus on scoring the mappings between question phrases and KB concepts from a candidate SPARQL query.
However, textual data can provide another angle on the problem, as question and answer entities are likely to be mentioned together somewhere in text passages.
For example, in the bottom right corner of Figure \ref{fig:model} we can see some passages that mention a pair of people, and the context of these mentions explains the nature of the relationships.
This data can be viewed as additional edges in a KB, that connect pairs of entities, and have associated language models, estimated from text phrases, that mention these entities.
Such edges do not have to coincide with the existing KB edges, and can connect arbitrary pairs of entities, that are mentioned together in text, therefore extending the KB.

We use the ClueWeb12 corpus with existing Freebase entity annotations and count different terms that occur in the context of a mention of a pair of different entities (we only consider mentions within 200 characters of each other).
To compute this unigram language model we take terms in between and 100 character before and after entity mentions.
A small sample of this data is presented in Table \ref{table:clueweb_entitypairs_langmodel}.

\begin{table}
\caption{Example of entity pairs along with the most popular terms mentioned around the entities}
\label{table:clueweb_entitypairs_langmodel}
\begin{tabular}{| p{1.25cm} | p{1.23cm} | p{4.5cm} |}
\hline
Entity 1 & Entity 2 & Term counts\\
\hline
John Edwards & Rielle Hunter & campaign, affair, mistress, child, former ...\\
\hline
John Edwards & Cate Edwards & daughter, former, senator, courthouse, greensboro, eldest ...\\
\hline
John Edwards & Elizabeth Edwards & wife, hunter, campaign, affair, cancer, rielle, husband ...\\
\hline
John Edwards & Frances Quinn & daughter, john, rielle, father, child, former, paternity...\\
\hline
\end{tabular}
\end{table}

We use this data to compute candidate ranking features in the following way.
Let us have question words $Q$ and an answer candidate, which contains a question entity $e_1$ and one or more answer entities $e_2$.
For each answer we compute a language model score:
$$p(Q|e_1, e_2) = \prod_{t\in Q} p(t | e_1, e_2)$$
and use minimum, average and maximum over all answer entities as features.
To address the sparsity problem, we again use embeddings, 
\ie for each entity pair a weighted (by counts) average embedding vector of terms is computed and minimum, average and maximum cosine similarities between these vectors and question token embeddings are used as features.

\subsection{Internal text data to enrich entity representation}
In addition to external text data, many knowledge bases, including Freebase, contain text data as well, \eg Freebase includes a description paragraph from Wikipedia for many of its entities.
These text fragments provide a general description of entities, which may include information relevant to the question, which was found useful for Text-QA \cite{Sun:2015:ODQ:2736277.2741651}.
For completeness, we include them in our system as well.
Each entity description is represented by a vector of tokens, and a vector of mentioned entities.
We compute cosine similarities between token and entity vectors of the question and description of each of the answer, and use minimum, average and maximum of the scores as features.
% In future work, we could explore incorporating any other entity profile text, such as full Wikipedia article.


\section{Evaluation}

We will evaluate on WebQuestions dataset, presented in \cite{Berant:EMNLP13}.
Hopefully we will show state of the art performance.

I will insert all results on WebQuestions from Codalab: https://worksheets.codalab.org/worksheets/0xba659fe363cb46e7a505c5b6a774dc8a/


I also need to get text-only baseline (something similar to AskMSR++).

\subsection{Ablation Study}

We need to run our system without each of the major components and report results.



\section{Analysis}

% IDEA1: We can also make evaluation on queries where correct predicate was never seen during training. Hopefully this will show that for such queries existing approach give worse result and we somehow improve it.

% IDEA2: Hypothesis: it's easier to find mentions of the named entities rather than more abstract ones, like profession.
% Therefore, maybe errors of our model will be more on these cases?

We also manually worked though wins and losses of Text2KB compared to the baseline system.
Below we provide some examples, that demonstrate advantages and weaknesses of our approach.

The first set of improvements come from the date range filter template, \eg for the question \textit{``who is the current leader of france 2010?''} our system returns a single correct result \textit{``Nicolas Sarkozy''} instead of the list of all French presidents.
The type model score feature helped in some cases, where there is a clear indication of the type of entity, expected as the answer, \eg \textit{``which state did anne hutchinson found?''} - \textit{``Rhode Island''}.

There are a number of cases when question entity identification using web search results helped to find the right KB entity, which wasn't detected from the question text only, \eg the question \textit{``what did romo do?''} mentions only the last name of the Dallas Cowboys quarterback, whereas web search results mentions the full name multiple times.
However, there are cases when additional entities actually made the system to return an incorrect answer, \eg for the question \textit{``what was lucille ball?''} Text2KB added the entity \textit{``I Love Lucy''}, and the candidate answer seeded from this entity got selected as the answer.
We should note, that we used a simple entity linking algorithms and strategy to extend question entities, namely we extend the question entity list if term from web search results entity name have high similarity to a term in the question.
A better strategies would probably fix the above mentioned problem.

Finally, below are some examples, improved by the proposed web search results features.
For the question \textit{``what did bruce jenner win gold medal for?''} the baseline system answered \textit{``1976 Summer Olympics''}, but web search results mention decathlon many times and thus Text2KB was able to rerank the candidates and return the entity \textit{``Athletics at the 1976 Summer Olympics - Men's Decathlon''}\footnote{Unfortunately, the entity selected as the answer during labeling is \textit{``Decathlon Challenge''}, which is a book Bruce Jenner wrote}.
Another interesting example is the question \textit{``what ship did darwin sail around the world?''}, which actually is hard to answer correctly because the ship entity is connected to the Charles Darwin entity through the \texttt{user.lindenb.default\_domain.scientist.known\_for} predicate along with some other entities like \textit{``Natural selection''}.
There is no predicate, and therefore no such RDF triple, that tells directly what kind of ship did Charles Darwin use.
We will see later, that in WebQuestions dataset there is a relatively big number of questions don't have a good match among the predicates.
Nevertheless, the name of the ship \textit{HMS Beagle} is mentioned multiple times in the web search results, and entity pair model computed from ClueWeb also has high scores for the terms ship and world, which gave Text2KB enough signal to answer with the ship (along with 2 other unrelated entities also related to \textit{``Charles Darwin''} through the same predicate).

We also found some cases, when our text-based features hurt the performance.
For example, the baseline system answered the question \textit{``when did tony romo got drafted?''} correctly, but since almost every mention of \textit{Tony Romo} follows with \textit{Dallas Cowboys}, Text2KB reranked the team name higher and returned it as the answer.

WebQuestions dataset isn't free of noise and quite a few answers are actually incorrect for various reasons.
When labeling the question ``what team does jordan own?'' mechanical turk workers had to select the answer from the page, corresponding to the country and not \textit{Michael Jordan} the basketball player.

\subsection{Error analysis}

Present extensive error analysis of questions that system doesn't get right.

Give results after case is fixed.


\section{Related Work}

% Recent development of large scale knowledge bases (e.g. dbPedia \cite{auer2007dbpedia}) and Freebase \cite{Bollacker:2008:FCC:1376616.1376746}) motivated research in open domain question answering over linked data.
In 2011 a series of QALD (Question Answering over Linked Data) evaluation campaigns has started.
You can find the most recent report in \cite{UngerFLNCCW15}.
These benchmarks use dbPedia knowledge base and usually provide a training set of questions, annotated with the ground truth SPARQL queries.
In QALD-3 a multilingual task has been introduced, and since QALD-4 the hybrid task is included.
This task asks participants to use both structured data and free form text available in dbPedia abstracts.
The formulation of the hybrid task is the most relevant to our work, but there are a couple of key differences.
Questions in the hybrid track are manually created in such a way, that they can \textit{only} be answered using a combination of RDF and free text data.
% whereas WebQuestions dataset contains a more realistic set of questions, which doesn't require any text data
Secondly, the hybrid task focuses on text data already present in a KB, whereas we are exploring external text resources.
In general, because of the expensive labeling process, QALD datasets are rather small, for example, QALD-5 training set for multilingual question answering includes 300 examples and 40 examples for the hybrid task.
The evaluation was performed on 50 questions for multilingual task and just 10 for hybrid.
Therefore, due to the scale of datasets and slightly different focus of tasks, we didn't attempt to evaluate our techniques on QALD benchmarks, but intend to explore it further in the future.

WebQuestions benchmark was introduced in \cite{Berant:EMNLP13}.
% and the approaches proposed since are usually divided into semantic parsing \cite{Berant:EMNLP13,berant2014semantic,berant2015imitation} and information extraction \cite{yao2014information,yao-scratch-qa-naacl2015,yao2014freebase,yih2015semantic,yu2014deep} based approaches depending on whether the system build a semantic representation of the question utterance or just use string matching to rank answers.
The proposed approaches differ in the algorithms used for various components, and, what is more relevant to our work, the use of external datasets.
To account for different ways a question can be formulated \cite{berant2014semantic} used a dataset of question clusters from WikiAnswers to learn a question paraphrasing model.
Another approach to learn term-predicate mapping is to use distant supervision \cite{mintz2009distant} to label a large text corpus, such as ClueWeb \cite{yao2014information}.
In this work we build on this idea and instead of focusing on term-predicate mappings, which might be too general, consider particular entity pairs.
Freebase RDF triples can automatically converted to questions using entity and predicate names \cite{BordesCW14:emnlp}.
Finally, many systems work with distributed vector representations for words and RDF triples and use various deep learning techniques for answer selection \cite{BordesCW14:emnlp,yih2015semantic}.
In all of these works, external resources are used to train a lexicon for matching questions to particular KB queries.
The use of external resources in this work is different, we are targeting better candidate generation and ranking by considering the actual answer entities rather than predicates used to extract them.

In general, combining different data sources, such as text documents and knowledge bases, for question answering is not a novel idea, and it has been already implemented in hybrid QA systems \cite{baudivs2015modeling,Barker12}.
Such systems typically have different pipelines that generate answer candidates from each of the data sources independently, and merge them to select the final answer at the end.
We make a step towards integration of approaches, by incorporating text resources into different stages of knowledge base question answering process.
This is similar to the work of \cite{Sun:2015:ODQ:2736277.2741651}, who explored the use of entity types and descriptions from a KB for text-based question answering, and \cite{dalton2014entity} explored such semantic annotations for ad-hoc document retrieval.

We should also mention OpenIE \cite{fader2011identifying}, which represent an interesting mixture between text and structured data.
Such knowledge repositories can be queried using structured query languages, and at the same time allows keyword matching against entities and predicates \cite{Fader:2014:OQA:2623330.2623677}.
% One can easily transform an existing KB to such a form by replacing predicates and entities with their names.
% This approach was losing to approaches based on a structured KB on WebQuestions, but had a better performance on a more general TREC QA and WikiAnswers datasets \cite{Fader:2014:OQA:2623330.2623677}.
In this work, we are borrowing an idea of learning about entity relationship via natural language phrases connecting them.
However, since we don't need to extract clean set of relation tuples, we can keep all kinds of phrases, mentioned around entity pairs.

\section{Conclusions and Future Work}
The results are obviously good and the paper should be accepted.

We leave joint hybrid question answering for future work.
If I don't have time and don't do Wikipedia profile pages in time, then this is also future work.

%ACKNOWLEDGMENTS are optional
\section{Acknowledgments}
This section is optional; it is a location for you
to acknowledge grants, funding, editing assistance and
what have you.  In the present case, for example, the
authors would like to thank Gerald Murray of ACM for
his help in codifying this \textit{Author's Guide}
and the \textbf{.cls} and \textbf{.tex} files that it describes.

%
% The following two commands are all you need in the
% initial runs of your .tex file to
% produce the bibliography for the citations in your paper.
\bibliographystyle{abbrv}
\bibliography{sigproc}  % sigproc.bib is the name of the Bibliography in this case
% You must have a proper ".bib" file
%  and remember to run:
% latex bibtex latex latex
% to resolve all references
%
% ACM needs 'a single self-contained file'!
%
%APPENDICES are optional
%\balancecolumns
% \appendix
%Appendix A

\end{document}
