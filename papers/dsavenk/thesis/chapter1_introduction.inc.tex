\chapter{Introduction}
\setcounter{page}{1}

It has long been a dream to communicate with a computer as one might with another human being using natural language.
We are now closer to this dream, as natural language interfaces become increasingly popular.
Our phones are already reasonably good at recognizing speech, and personal assistants, such as Apple Siri, Google Assistant, Microsoft Cortana, Amazon Alexa, \etc, help us with everyday tasks and answer some of our questions.
Chat bots are arguably considered ``the next big thing'', and a number of startups developing this kind of technology have emerged in Silicon Valley and around the world\footnote{\href{url}{http://time.com/4194063/chatbots-facebook-messenger-kik-wechat/}}.

Question answering is one of the major components of such personal assistants.
Existing techniques already allow users to get direct answers to some of their questions.
However, by some estimates\footnote{\href{url}{https://www.stonetemple.com/the-growth-of-rich-answers-in-googles-search-results/}} for $\sim 70\%$ of more complex questions users still have to dig into the ``10 blue links'' from search engine results page and extract or synthesize answers from information buried within the retrieved documents.
In order to make a shift towards more intelligent personal assistants, this gap needs to be closed.
Therefore, in my dissertation, I focus on helping users solve their informational needs by improving question answering methods and techniques.

User questions vary in many different aspects, and each type has its own set of challenges~\cite{voorhees2001trec,yang2016beyond}.
It is common to divide questions into \textit{factoid} and \textit{non-factoid}.
Factoid questions are inquiring about certain facts and can be answered with entity names or attributes, such as dates or numbers.
An example of a factoid question is ``\textit{What book did John Steinbeck wrote about the people in the dust bowl?}'' (answer: ``\textit{The Grapes of Wrath}'').
Of course, there is a variety of questions, that do not fall into this category, \eg how-to, ``why'' questions, recommendation and opinion questions, \etc
The literature usually refers to all such questions as ``\textit{non-factoid}''.
An example of such question is ``\textit{Why did John Steinbeck name his book the Grapes of Wrath?}''.
Most of the research in automatic question answering focused on factoid questions \cite{BerantCFL13:sempre,Cafarella:2008:WEP:1453856.1453916,voorhees2001trec,lin2007exploration}, but relatively recently more and more works have started targeting the category of non-factoid questions~\cite{overviewliveqa15,cohen2016end,fried2015higher,surdeanu2011learning,sharp2015spinning,tymoshenko2016learning,yang2016beyond}.

\begin{table}[h!t]
	\centering
	\small
	\begin{tabular}{p{2cm}|p{5.3cm}|p{5.3cm}}
		& Unstructured data & Structured data \\
		\hline
		Factoid questions & \multicolumn{1}{|c|}{Text} & \multicolumn{1}{|c}{Knowledge Bases} \\
		& + easy to match against question text & + aggregate all the information about entities\\
		& + cover a variety of different information types & allow complex queries over this data using special languages (e.g. SPARQL) \\
		& - each text phrase encodes a limited amount of information about mentioned entities & - hard to translate natural language questions into special query languages \\
		&  & - KBs are incomplete (missing entities, facts and properties) \\
		\hline
		Non-factoid questions & \multicolumn{1}{|c|}{Text} & \multicolumn{1}{|c}{Question-Answer pairs} \\
		& + contain relevant information to a big fraction of user needs & + easy to find a relevant answer by matching the corresponding questions \\
		& - hard to extract semantic meaning of text and match it against the question (lexical gap) & - cover a smaller subset of user information needs \\
	\end{tabular}
	\caption{Pros and cons of structured and unstructured data sources for factoid and non-factoid question answering.}
	\label{table:intro:data_procons}
\end{table}

More than 3.5 billion people in the world have access to the internet, and this number has increased tenfold from 1999 to 2013~\footnote{http://www.internetlivestats.com/}.
Over the years, people have contributed a vast amount of data, and created many highly valuable resources, such as encyclopedias (Wikipedia, WedMD, Britannica), community question answering websites (Quora, Yahoo!~Answers, Answers.com), social media (Twitter, Facebook, Snapchat), curated knowledge bases (Freebase, WikiData, DBpedia, YAGO), and others.
With more than a billion websites, the internet stores a huge volume of information, which could be useful to answer user questions.
By their nature, data sources can be classified into \textit{unstructured} (\eg raw natural language text), \textit{semi-structured} (\eg tables or question-answer pairs) and \textit{structured} (\eg knowledge bases).
Each of the formats offers a unique set of features, which have their advantages and limitations for question answering, and can often complement each other (Table~\ref{table:intro:data_procons}).
%Structured data is usually easier to query and reason with, but most of the business data is stored in the unstructured format, such as raw text~\cite{grimes2005structure}.
%However, structuring the data adds limitations, which often reduce the coverage of such data sources.
A number of methods have been proposed for question answering using text collections, knowledge bases (KB) or archives of question-answer (QnA) pairs.
Most of the developed systems use either a single source of data or combine multiple independent pipelines, each of which operates over a separate data source.

Two major paradigms for factoid question answering are knowledge base question answering (KBQA) and text-based question answer (TextQA).
The information contained in a huge volume of text data can be relatively easily queried using terms and phrases from the original question in order to retrieve sentences that might contain the answer.
However, each sentence encodes only limited amount of information about mentioned entities and aggregating it over unstructured data is quite problematic.

On the other hand, modern large scale knowledge bases, such as Freebase \cite{Bollacker:2008:FCC:1376616.1376746}, dbPedia \cite{auer2007dbpedia}, YAGO \cite{yago3}, WikiData \cite{vrandevcic2014wikidata}, accumulate information about millions of entities into a graph of \texttt{[subject, predicate, object]} RDF triples.
One of the issues with KBs is that they are inherently incomplete and miss a lot of entities, facts, and predicates~\cite{Dong:2014:KVW:2623330.2623623}.
Some of these issues are addressed by relation extraction techniques, which can derive factual knowledge from raw text data~\cite{MintzBSJ09}, web tables~\cite{Cafarella:2008:WEP:1453856.1453916} or pages~\cite{Cafarella:2009:WES:1519103.1519112}.
In my dissertation, I develop a method to further extend this set of sources to question-answer pairs, available in abundance on community question answering platforms.
This technique allows us to extract additional information, which may not be easily accessible in other sources.

While extremely useful, relation extraction techniques are not perfect and introduce both precision and recall errors.
An alternative approach is to use raw text and knowledge base data together for joint reasoning.
However, mapping between natural languages phrases and knowledge base concepts is not trivial, and is traditionally done by building huge lexicons~\cite{BerantCFL13:sempre,BerantL14:parasempre}.
The \textit{Text2KB} model, which I developed in my Ph.D. work, takes a different approach, and exploits techniques from text-based question answering to improve different stages of the KBQA process at run-time, \ie by retrieving relevant pieces of information from unstructured and semi-structured data sources, and using them to help KBQA system to score candidate answers.

Unfortunately, not all user questions align well with a knowledge base schema, even with large ones, such as Freebase~\cite{Fader:2014:OQA:2623330.2623677}.
The \textit{EviNets} model I developed in my dissertation, is a memory-augmented neural network architecture, which aggregates evidence in support for answer candidates from multiple different data sources, such as RDF triples and text sentences.
The proposed approach improves the performance over both Text and KB-based question answering, by better utilization of all available information.

% THIS PIECE IS GOOD, BUT IT DUPLICATES SOMETHING I HAVE ALREADY SAID. KEEPING IT JUST IN CASE...
% Billions of documents on the web contain all kinds of knowledge about the world, which can be retrieved to answer user questions.
%However, each individual statement includes a very limited amount of information about mentioned entities.
%On the other side, modern open domain large-scale knowledge bases, such as dbPedia\footnote{http://wiki.dbpedia.org/}, YAGO\cite{yago3}, Freebase\footnote{http://www.freebase.com}, WikiData\footnote{https://www.wikidata.org/}, etc., contain millions of entities and facts about them and are quite effective in answering some of the user questions.
%However, knowledge bases have their own disadvantages:
%\begin{itemize}
%\item knowledge bases are inherently incomplete \cite{Dong:2014:KVW:2623330.2623623}, even the largest existing resources miss a lot of entities, facts, and properties, that might be of interest to some users.
%\item it is quite challenging to translate a natural language question into a structured language, such as SPARQL, to query a knowledge base \cite{BerantCFL13:sempre}.
%\end{itemize}

In non-factoid question answering, one of the main challenges is the diversity of question and answer types.
Reusing answers from previously posted similar questions, which could be found, for example, in CQA archives, was demonstrated to be quite effective to answer new questions \cite{carmel2000eresponder,Shtok:2012:LPA:2187836.2187939}.
Unfortunately, it is not always possible to find a similar question, that has already been answered, because many information needs are unique in general or in details.
Alternative strategies include ranking text passages extracted from retrieved web documents by estimating semantic similarity between the question and an answer candidate \cite{soricut2006automatic}.
TREC LiveQA shared task\footnote{\href{url}{http://trec-liveqa.org}}, started in 2015 to advance research in complex question answering, asks participants to build a system to answer real user questions in real-time using any information available.
In this dissertation, I developed an open source state-of-the-art system, which utilizes both CQA archives and web search data sources, and combines them in a single model.

Unfortunately, no matter how good a QA system is, there likely to be cases when it is unable to return a satisfactory response to a user question, \eg existing data sources might not contain the necessary information, or a system might fail to rank a good answer on top of others.
Such failures can be detrimental to the overall user experience with a QA system.
One way to mitigate this challenging situation is to put a human in the loop, \eg let a system consult a group of workers, who can provide some kind of feedback and help return a more satisfactory answer.
\textit{EmoryCRQA} extends my automatic non-factoid question answering system with a real-time crowdsourcing module.
Crowdsourcing serves as an additional source of answer candidates and their quality ratings, which are used for re-ranking to return a better final response.
This system achieved the top performance on TREC LiveQA 2016 shared task.

The scenarios described above follow a traditional one-way interaction scenario, where a user issues a single query and a system needs to come up with a response.
However, modern conversational interfaces open many opportunities to enrich this scenario, and transition to information seeking dialogs, where a system may ask additional clarification questions, take feedback, \etc.
In Chapter~\ref{chapter:conversation} of my dissertation I focus on conversational question answering.
First, we conducted a user study with the goal to learn more about how users use dialogs, either with humans or chatbots, to solve informational needs.
The results of the user study suggest directions for future research in the area.
Next, we turn to two particular types of interactions, which a system can exploit to help users: search hints, designed to assist a user with complex multi-step tasks, and clarification questions, which may be asked to resolve certain ambiguities in user questions.
Together, our results and findings about conversational search provide a number of insights and ideas, which can be helpful for future research.

\clearpage

\section{Contributions}
\label{section:intro:contributions}

The main contributions of the dissertation are summarized as follows:
\begin{itemize}
	\item \textbf{Relation extraction from community generated question-answer pairs:} The dissertation develops a model for extracting factual knowledge for KB completion from CQA question-answer pairs.
	This method allows to extract more [subject, predicate, object] triples from available information and helps with knowledge base incompletion problem (Section~\ref{section:factoid:cqarelextract}).
	\item \textbf{Techniques for augmenting knowledge base question answering with unstructured text data:} The dissertation develops a novel Text2KB model for knowledge base question answering, which utilizes various available unstructured information, such as web search results, CQA archives, and annotated document collections, and improves the performance of a pure KBQA system (Section~\ref{section:factoid:text2kb}).
	\item \textbf{Framework for combining text and knowledge base evidence for factoid question answering:} The dissertation develops EviNets, a memory-augmented neural network architecture for aggregating multiple different pieces of information, as evidence in support of different answer candidates.
	To show the efficiency of the proposed architecture over a variety of user questions, I developed a new entity-centric factoid QA dataset, derived from the Yahoo!~Answers archive (Section~\ref{section:factoid:evinets}).
	\item \textbf{Non-factoid question answering system:} The dissertation develops a state-of-the-art system for non-factoid question answering, which showed very competitive results on both TREC LiveQA 2015 and 2016 shared tasks.
	The system combines vertical CQA and general web searches to build candidates from both retrieved question-answer pairs and web documents, which are then ranked with a trained learning-to-rank model (Section~\ref{section:non-factoid:system}).
	\item \textbf{Techniques for real-time crowdsourcing for question answering:} The dissertation proposed a method to incorporate crowdsourcing into a real-time question answering system. EmoryCRQA system, which extends our LiveQA approach with the crowdsourcing module to obtain additional and rate existing answer candidates, shows significant improvements over the baseline performance and achieves the top result on TREC LiveQA 2016 task (Section~\ref{section:non-factoid:crowdsourcing}).
	\item \textbf{Exploration of conversational question answering:} The dissertation investigates user perception and behavior patterns in dialog-based information seeking.
	We further study search hints and clarification questions, as particular examples of actions available to a QA system in dialog settings, and provide the results and implications of the analysis we conducted on the topics (Chapter~\ref{chapter:conversation}).
\end{itemize}

Together, the results presented in this Ph.D. dissertation push research in question answering forward by improving the key components in the QA pipeline with techniques to combine different unstructured, semi-structured and structured data source, and by exploring the possibilities of conversational interfaces for information seeking scenarios. 

%\section{Organization}
%\label{section:intro:organization}

%%In summary, in this thesis I describe several methods and techniques, that improve the performance of automatic question answering and describe our findings in a conversational search.
%The thesis is organized as follows: In Chapter~\ref{chapter:related}, related work is reviewed to put the thesis in context.
%Then, Chapter~\ref{chapter:factoid} presents ideas and models, that improve factoid question answering using both structured knowledge bases, semi-structured question-answer pairs and unstructured text data sources.
%Next, Chapter~\ref{chapter:non-factoid} focuses on non-factoid question answering, \ie user information needs, that cannot be answered with an entity or its attribute.
%I describe a state-of-the-art open source question answering system I developed to participate in TREC LiveQA evaluation campaign, which achieves high performances in both 2015 and 2016 tasks.
%Further, I describe \textit{EmoryCRQA}: an extension of the non-factoid QA system that integrates a real-time crowdsourcing module, which achieved the top result in TREC LiveQA 2016.
%Finally, Chapter~\ref{chapter:conversation} describes the findings of the user study we conducted to learn about user behavior in human-human and human-system conversation search scenarios.
%Additionally, it describes the analysis and experiments on strategic search hints and clarification questions, as two particular types of interactions, available to a QA system in dialog settings.
%Finally, in Chapter~\ref{chapter:conclusion}, the thesis is concluded with a summary of the findings, a discussion about result implications, limitations of the proposed techniques, and future research directions.
