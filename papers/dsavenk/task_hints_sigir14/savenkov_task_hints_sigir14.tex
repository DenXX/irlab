% This is "sig-alternate.tex" V2.0 May 2012
% This file should be compiled with V2.5 of "sig-alternate.cls" May 2012
%
% This example file demonstrates the use of the 'sig-alternate.cls'
% V2.5 LaTeX2e document class file. It is for those submitting
% articles to ACM Conference Proceedings WHO DO NOT WISH TO
% STRICTLY ADHERE TO THE SIGS (PUBS-BOARD-ENDORSED) STYLE.
% The 'sig-alternate.cls' file will produce a similar-looking,
% albeit, 'tighter' paper resulting in, invariably, fewer pages.
%
% ----------------------------------------------------------------------------------------------------------------
% This .tex file (and associated .cls V2.5) produces:
%       1) The Permission Statement
%       2) The Conference (location) Info information
%       3) The Copyright Line with ACM data
%       4) NO page numbers
%
% as against the acm_proc_article-sp.cls file which
% DOES NOT produce 1) thru' 3) above.
%
% Using 'sig-alternate.cls' you have control, however, from within
% the source .tex file, over both the CopyrightYear
% (defaulted to 200X) and the ACM Copyright Data
% (defaulted to X-XXXXX-XX-X/XX/XX).
% e.g.
% \CopyrightYear{2007} will cause 2007 to appear in the copyright line.
% \crdata{0-12345-67-8/90/12} will cause 0-12345-67-8/90/12 to appear in the copyright line.
%
% ---------------------------------------------------------------------------------------------------------------
% This .tex source is an example which *does* use
% the .bib file (from which the .bbl file % is produced).
% REMEMBER HOWEVER: After having produced the .bbl file,
% and prior to final submission, you *NEED* to 'insert'
% your .bbl file into your source .tex file so as to provide
% ONE 'self-contained' source file.
%
% ================= IF YOU HAVE QUESTIONS =======================
% Questions regarding the SIGS styles, SIGS policies and
% procedures, Conferences etc. should be sent to
% Adrienne Griscti (griscti@acm.org)
%
% Technical questions _only_ to
% Gerald Murray (murray@hq.acm.org)
% ===============================================================
%
% For tracking purposes - this is V2.0 - May 2012

\documentclass{sig-alternate}

\usepackage{xcolor}
\newcommand\todo[1]{\textcolor{red}{#1}}


\begin{document}
%
% --- Author Metadata here ---
\conferenceinfo{SIGIR}{'14 Gold Coast, Australia}
%\CopyrightYear{2007} % Allows default copyright year (20XX) to be over-ridden - IF NEED BE.
%\crdata{0-12345-67-8/90/01}  % Allows default copyright data (0-89791-88-6/97/05) to be over-ridden - IF NEED BE.
% --- End of Author Metadata ---

\title{On a Tip of Your Search: Evaluating Effect of Search Tips for Complex Informational Search Tasks}

%
% You need the command \numberofauthors to handle the 'placement
% and alignment' of the authors beneath the title.
%
% For aesthetic reasons, we recommend 'three authors at a time'
% i.e. three 'name/affiliation blocks' be placed beneath the title.
%
% NOTE: You are NOT restricted in how many 'rows' of
% "name/affiliations" may appear. We just ask that you restrict
% the number of 'columns' to three.
%
% Because of the available 'opening page real-estate'
% we ask you to refrain from putting more than six authors
% (two rows with three columns) beneath the article title.
% More than six makes the first-page appear very cluttered indeed.
%
% Use the \alignauthor commands to handle the names
% and affiliations for an 'aesthetic maximum' of six authors.
% Add names, affiliations, addresses for
% the seventh etc. author(s) as the argument for the
% \additionalauthors command.
% These 'additional authors' will be output/set for you
% without further effort on your part as the last section in
% the body of your article BEFORE References or any Appendices.

\numberofauthors{2} 

\author{
% 1st. author
\alignauthor
Denis Savenkov\\
       \affaddr{Emory University}\\
       \email{dsavenk@emory.edu}
% 2nd. author
\alignauthor
Eugene Agichtein\\
       \affaddr{Emory University}\\
       \email{eugene@mathcs.emory.edu}
}
\date{17 February 2014}
% Just remember to make sure that the TOTAL number of authors
% is the number that will appear on the first page PLUS the
% number that will appear in the \additionalauthors section.

\maketitle
\begin{abstract}
Search engine is a ubiquitous tool used by millions of people on a daily basis.
However, as with every tool, certain skills are required in order to use it efficiently.
Unfortunately users have different experience and not everybody is able to find answers to all questions she is interested in \todo{[find a paper to cite here]}.
Helping users to develop their search skills was included as one of the key research directions by \cite{Allan:2012:FCO:2215676.2215678}. However, the assistance offered by the modern search engines are limited to query suggestion and spelling correction \todo{[something else?]}.
A number of researches are available that study different ways of helping users be more successful with their searches \todo{[cite some reviews, or a couple of different papers]}. 
In this work we study the effect of showing users search tips designed to help asking the better queries when solving a difficult informational search task.
We show that ``the right'' tips can improve users' success rate.
However generic hints might be misleading and detrimental to user search experience.
\end{abstract}

% A category with the (minimum) three required fields
\category{H.3.3}{Information storage and retrieval}{Information Search and Retrieval}[query formulation, search process]
%A category including the fourth, optional field follows...

\terms{Measurement, Design, Experimentation, Human Factors}

\keywords{User studies, search interface, experimental design, effectiveness measures, query reformulation, expertise, tactics, tips, suggestions, assistance, efficiency.}

\section{Introduction}
\todo{Motivation of search hints as alternative/addition to query suggestion, especially for tasks when search cannot be solved with a single query}

Improving user search experience is usually considered as a problem of improving retrieval performance. 

\todo{Query suggestion is similar to tips, but is not guaranteed to be good, usually rely on more behavior data available for the new query}

\todo{Query terms selection}

\todo{Other dialog systems?}

\todo{Showing predicted retrieval performance}

\todo{Search snippets}


\section{Related Work}
\todo{Based on related work review from Daniel Russel \cite{Moraveji:2011:MIU:2009916.2009966}}.

\todo{Summarize works on task level query suggestion}
Experiment Design
\todo{State main difference from Dan Russel paper: we focus on informational tasks which are solved with web search and hints are less likely to be 100\% helpful}

\section{Tips for Difficult Search Tasks}

\subsection{Web Search Game}
\todo{First part is about the uFindIt game with screenshot}
To estimate the effect of search tips on user behavior and success we used the uFindIt game, proposed in \cite{Ageev:2011:FYG:2009916.2009965}. The goal of the game is to find the answers to given search tasks using the provided web search interface. 

\subsection{Search Tasks Description}
\todo{Describe and give text of search tasks used in the study, mention that they came from agoogleaday}

\subsection{Experiment design}
\todo{describe the setup, groups and search tips given}

\section{Results}
\todo{Provide raw results of the experiments, e.g. how many participants, how many HITs accepted, rejected, problems. Show examples of search trails for successful and unsuccessful searches}.

\subsection{Analysis}
\todo{Give a table and a couple of pictures with main quantitative results}

The main findings can be summarized as follows:
\begin{itemize}
\item ``Correct'' search tips allows users to find correct answer more often and do this faster than without search tips
\item ``General'' search hints can have detrimental effect on search success, reducing the success rate and increasing the time spent on task
\end{itemize}

\section{Discussions and Future Work}
\todo{Make a conclusion by summarizing the findings one more time}

\todo{Speculate on the negative effect of general search hints - distracting? hard to follow?, less satisfaction when tips were shown - self satisfaction?}. 

%ACKNOWLEDGMENTS are optional
\section{Acknowledgments}
Thanks to Dan Russel for sharing questions.

%
% The following two commands are all you need in the
% initial runs of your .tex file to
% produce the bibliography for the citations in your paper.
\bibliographystyle{abbrv}
\bibliography{sigproc}  % sigproc.bib is the name of the Bibliography in this case
% You must have a proper ".bib" file
%  and remember to run:
% latex bibtex latex latex
% to resolve all references
%
% ACM needs 'a single self-contained file'!
%
%APPENDICES are optional
%\balancecolumns

\end{document}
