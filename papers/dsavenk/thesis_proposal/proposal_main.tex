% thesis_main.tex
%
% You don't need to change this file unless you use a different number
% of chapters than the template. See the EDIT HERE line below.
%
% 
%\documentclass[12pt,dblspace]{report}
\documentclass[12pt]{report}

\newcommand{\comment}[1]{}
\newcommand{\YA}{Yahoo!\ Answers}
\newcommand{\eg}[0]{{\em e.g. }}
\newcommand{\etc}[0]{{\em etc.}}
\newcommand{\ie}[0]{{\em i.e. }}
\newcommand{\wex}[1]{`{\em #1}'}
\newcommand{\shortcite}[1]{\cite{#1}}
%\usepackage{url}
\usepackage[hyphens]{url}
% \usepackage{subfigure}
\usepackage{multirow}
\usepackage[show]{chato-notes}
\usepackage{pdfpages}

\usepackage{natbib}
\usepackage[nottoc]{tocbibind}
\usepackage{datetime}
\usepackage{caption}
\usepackage{subcaption}

%\usepackage{geometry} 
\usepackage[margin=.8in]{geometry}
\usepackage{graphicx}
\usepackage{amssymb}
\usepackage{amsmath}
\usepackage{epstopdf}
\usepackage{setspace}
\usepackage{listings}
\usepackage{amsthm}
\usepackage{url}
\usepackage{float}
\usepackage{hyperref}

\floatstyle{plain} 
%\floatstyle{boxed}
\restylefloat{figure}

\DeclareGraphicsRule{.tif}{png}{.png}{`convert #1 `dirname #1`/`basename #1 .tif`.png}
\hsize=5in
\vsize=7.5in
%\hoffset=.05in
\voffset=.0in
%\hoffset=.25in
%\voffset=.25in
\renewcommand{\baselinestretch}{1}
%\renewcommand{\baselinestretch}{2}

%
% To get page numbering exactly right (not really needed),
% Uncomment next line after editing toc and lof files appropriately.
%\nofiles
%





\newcommand{\mychapter}[1]{\newpage \vspace*{0.00mm} \refstepcounter{chapter} 
	
	{\LARGE \bf  \noindent \thechapter \hspace*{0.5em}   
		#1\baselineskip=1.0\normalbaselineskip\par}
	
			\vspace*{3ex} \par 
\addcontentsline{toc}{chapter}{\protect \numberline{\thechapter}{#1}}   }


%\newcommand{\mychapter}[1]{\newpage \vspace*{0.01mm} \refstepcounter{chapter} 
%	\begin{center}
%	{\huge \bf Chapter \space  \thechapter \vspace*{1em}  \par 
%		#1\baselineskip=1.0\normalbaselineskip\par}
%		 \end{center} 
%			\vspace*{3ex} \par 
%\addcontentsline{toc}{chapter}{\protect \numberline{\thechapter}{#1}}   }


%\newcommand{\myappendix}[1]{\newpage \vspace*{0.01mm} \refstepcounter{chapter}
%        \begin{center}
%        {\LARGE \bf Appendix \space  \thechapter \vspace*{1em}  \par
%                #1\baselineskip=1.0\normalbaselineskip\par}
%                 \end{center}
%                        \vspace*{3ex} \par
%\addcontentsline{toc}{chapter}{Appendix \protect \numberline{\thechapter} - {#1} } }

\pagenumbering{arabic}
\setcounter{page}{1}
\pagestyle{myheadings}
\begin{document}
\renewcommand{\baselinestretch}{1.3}
%\renewcommand{\baselinestretch}{1.3}

\input epsf

\setlength{\headsep}{0.15in}
%\setlength{\headsep}{1.15in}
\setlength{\topmargin}{-.5in}
\pagenumbering{roman}
\pagestyle{empty}



\title{
\textbf{Question Answering Using Structured and Unstructured Data} \\
%\textbf{Improving Information Access with Community Question Answering} \\
\normalfont Doctoral thesis proposal}
\author{\textbf{Denis Savenkov}\\
      Dept. of Math \& Computer Science\\
      Emory University\\
      denis.savenkov@emory.edu
}


\newdateformat{mydate}{\monthname[\THEMONTH], \THEYEAR}

\mydate

\maketitle


\begin{abstract}

Over more than half a century of research, the area of automatic Question Answering (QA) has progressed from small single domain systems to IBM Watson, who defeated best human competitors in the Jeopardy! TV show \cite{ferrucci2010building}.
However, many of our questions are still left unanswered, and we still have a lot to do to move beyond 10 blue links in search results \cite{etzioni2011search} as for most of the questions users still have to dig into the retrieved documents or post questions to the community question answering (CQA) websites.
Questions come in different flavors, some are asking about a certain fact and can be answered with a short phrase, such as entity name, date or number.
Such questions are typically referred to as \textit{factoid}, as opposed to rest of the questions, which are often called \textit{non-factoid}.
In my thesis I focus on three topics in QA: 1). combination of structured and unstructured data to improve factoid question answering; 2). improving question summarization, candidate scoring and answer generation using recent advances in neural network research and better utilization of source web document structure; 3). interactions between a question answering system and real people.

Text document collections and knowledge bases (KB) are very effective in answering certain types of factoid questions, but they are also complimentary to each other.
I propose to combine these data sources via semantic annotation of KB entity mentions, which effectively extends the knowledge base with additional unstructured information, often missing or complimentary to the KB data.

Non-factoid question answering is somewhat harder as it deals with a more diverse set of question and answer types.
In my thesis I propose to improve performance of different stages of QA system pipeline by better utilization of the structure of a web page where a candidate answer is extracted from, and using deep learning techniques, inspired by recent successes in machine translation \cite{bahdanau2014neural}, text summarization \cite{rush-chopra-weston:2015:EMNLP}, automatic caption generation \cite{karpathy2015deep} and answer sentence scoring \cite{WangN15}.

The focus of the last part of the thesis is on the interaction between human and a search or question answering system.
Unfortunately, there always be cases, when a machine cannot provide a good answer to the question.
In such cases, a QA system may come back to the user with some suggestions on how a user can solve his search problem, or with a clarification question aiming at resolving certain ambiguities.
Alternatively, a machine can consult with the crowd in order to get the answer or help it decide on certain alternatives.

In summary, the goal of the proposed research is to improve the performance of question answering over a variety of different questions a user might have, and to study some reply strategies in case a system fails to deliver a good response.
I believe, that results of the proposed work will be useful for the future research in improving automatic question answering.



% THIS PART IS OLDER...

%Over the year of research most efforts were put on factoid questions, which can be answered with a short phrase, \eg an entity name, date, number, etc.
%Modern QA systems employ a variety of different unstructured (text-corpora), semi-structured (tables, Wikipedia infoboxes, question-answer pairs) and structured (databases, knowledge bases) data sources to generate candidate answers.
%Each of the data sources has its own advantages and limitations, in particular a text fragment encodes very limited amount of information about the entities involved in the statements, which complicates the reasoning about the answer correctness.
%For example, most factoid QA systems tries to substitute missing information with a prediction, \ie predict an expected lexical answer type (LAT) from the question and match it against the also predicted answer entity type.
%On the other side of the spectrum knowledge bases (KB) aggregate all available information about entities and support effective querying with a structured query language, such as SPARQL.
%The problem comes when we need to translate natural language information need to a structured query.
%Modern knowledge base question answering (KBQA) systems use question-answer pairs (QnA), question paraphrases and other resources to learn a lexicon to map from natural language phrases to knowledge base objects, which is still limited and works well for relatively popular simple questions.
%In addition knowledge bases are inherently incomplete and many entities, predicates and facts are simply missing.
%Therefore, it make sense to combine different data sources for question answering, and this approach was already shown to be successful by systems such as IBM Watson, but they treat different data sources mostly independently and use them to produce as a set of candidates, which are then ranked and the best answer is selected.
% However, for many questions it might be hard to find good candidates in the first place, and one would benefit from utilizing all available resources together at this stage.
%In my dissertation I propose to consider unstructured textual and structured knowledge base resources, connected via entity linking, together for joint reasoning on the candidate generation stage.
%Existing datasets for question answering are either relatively small (QALD tasks), focused on text (TREC QA) or on knowledge bases only (\eg WebQuestions).
%To evaluate the approach I'm going to build a new realistic dataset extracted from Yahoo! Answers question-answer pairs.

%Beyond factoid questions we have a plethora of different information needs, that require more than a simple fact to answer.
%Such questions are usually called non-factoid and more and more research effort is devoted to answering such questions.
%In 2015 Text REtrieval Conference (TREC) pioneered LiveQA shared task track, which targets automatic question answering of various types of questions user post on Yahoo! Answers Community Question Answering (CQA) website.
%Existing research has demonstrated the effectiveness of reusing answers to similar previously posted questions, but in many cases such questions are not available or challenging to find.
%Alternatively, existing systems rank passages extracted from regular web pages.
%However, ranking is complicated due to the lexical gap between question and answer text.
%Knowledge about what question does a paragraph of text answers would be very useful signal for ranking, which is supported by the results of the winning TREC LiveQA approach.
%In my thesis I propose to make a step further and automatically extract candidates text passages along with questions which they answers.
%This can be done by automatically detecting question-answer pairs from certain web pages (\eg forums, FAQ, \etc).
%In addition, we can build upon the recent success with automatic text generation by recurrent neural networks and train a model to predict a question for a given text fragment.

%In summary, this dissertation aims to improve the performance of automatic question answering systems for both factoid and non-factoid question answering.

\end{abstract}


\tableofcontents
%\listoffigures
%\listoftables




%%%%%%%%%%%%%%%%%%%%%%%%% EDIT HERE %%%%%%%%%%%%%%%%%%%%%%%%%%%%%%%%
% Change these lines to adjust for different numbers of chapters /
% appendices
%%%%%%%%%%%%%%%%%%%%%%%%%%%%%%%%%%%%%%%%%%%%%%%%%%%%%%%%%%%%%%%%%%%%
% chap1.tex
%
% First chapter file is different from others
%
\mychapter{Introduction}
\label{chapter:intro}

\pagenumbering{arabic}
\setcounter{page}{1}
\pagestyle{myheadings}

%Uncomment to switch spacing.
%\baselineskip=5px

\newtheorem{definition}{Definition}
\newtheorem{proposition}{Proposition}


\section{Motivation}

It has long been a dream to communicate with a computer as one might with another human being using natural language speech and text.
Nowadays, we are coming closer to this dream, as natural language interfaces become increasingly popular.
Our phones are already reasonably good at recognizing speech, and personal assistants, such as Apple Siri, Google Now, Microsoft Cortana, Amazon Alexa, etc., help us with everyday tasks and answer some of our questions.
Chat bots are arguably considered ``the next big thing'', and a number of startups developing this kind of technology has emerged in Silicon Valley and around the world\footnote{http://time.com/4194063/chatbots-facebook-messenger-kik-wechat/}.

Question answering is one of the major components of such personal assistants.
Existing techniques already allow users to get direct answers to some of their questions.
However, by some estimates\footnote{https://www.stonetemple.com/the-growth-of-rich-answers-in-googles-search-results/} for $\sim$ 70\% of more complex questions users still have to dig into the ``10 blue links'' and extract or synthesize answers from information buried within the retrieved documents.
In order to make a shift towards more intelligent personal assistants this gap needs to be closed.
Therefore, in my thesis I focus on helping users get answers to their questions by improving question answering methods and the ways a system interact with its users.

User questions vary in many different aspects, each of which has its own set of challenges.
It's common to divide questions into \textit{factoid} and \textit{non-factoid}.
Factoid questions are inquiring about certain facts and can be answered by a short phrase (or list), \ie entity name, date or number.
An example of a factoid question is ``\textit{What book did John Steinbeck wrote about the people in the dust bowl?}'' (answer: ``\textit{The Grapes of Wrath}'').
Of course, there is a variety of questions, that do not fall into this group, \eg how-to and why questions, recommendation and opinion questions, \etc.
The literature usually refers to all these questions by ``non-factoid questions'' umbrella term.
Most of the research in automatic question answering focused on factoid questions \cite{voorhees2001trec}, and recently more and more works target often more complex non-factoid category \cite{overviewliveqa15}.
These types of questions provide quite distinct set of challenges and methods applied to them are often quite different, therefore in my thesis I will first study factoid QA and then propose some ideas to improve non-factoid QA.

Automated question answering systems use various data sources to generate answers to user questions.
By their nature, data sources can be classified into \textit{unstructured} (\eg raw natural language text), \textit{semi-structured} (\eg tables) and \textit{structured} (\eg knowledge bases).
Each of these types of data has certain advantages and limitations (Table \ref{table:data_procons}).
There are a number of methods designed for question answering using text collections, knowledge bases or archives of question-answer (QnA) pairs.
Most of the developed systems use either a single source of data, or combine multiple independent pipelines, each of which operates over a separate data source.
Motivated by the fact that advantages and disadvantages of structured and unstructured data sources complement each other, In my thesis I propose to study methods of combining different data sources for joint reasoning for factoid and non-factoid questions.

\begin{table}
\centering
\caption{Pros and cons of structured and unstructured data sources for factoid and non-factoid question answering}
\begin{tabular}{| l | p{6cm} | p{6cm} |}
\hline
 & unstructured data & structured data \\
\hline
factoid questions & \multicolumn{1}{|c|}{Text} & \multicolumn{1}{|c|}{Knowledge Bases} \\
 & + easy to match against question text & + aggregate all the information about entities\\
 & + cover a variety of different information types & allow complex queries over this data using special languages (e.g. SPARQL) \\
 & - each text phrase encodes a limited amount of information about mentioned entities & - hard to translate natural language questions into special query languages \\
&  & - KBs are incomplete (missing entities, facts and properties) \\
\hline
non-factoid questions & \multicolumn{1}{|c|}{Text} & \multicolumn{1}{|c|}{Question-Answer pairs} \\
 & + contain relevant information to a big chunk of user needs & + easy to find a relevant answer by matching the corresponding questions \\
 & - hard to extract semantic meaning of a paragraph to match against the question (lexical gap) & - cover a smaller subset of user information needs \\
\hline
\end{tabular}
\label{table:data_procons}
\end{table}

Two major paradigms for factoid question answering are knowledge base question answering (KBQA) and text-based question answer (TextQA).
Information contained in a huge volume of text data on the web can be relatively easily queried using terms and phrases from the original question in order to retrieve sentences that might contain the answer.
However, each sentence encode very limited amount of information about mentioned entities and aggregating it over unstructured data is quite problematic.
On the other hand, modern large scale knowledge bases, such as Freebase \cite{Bollacker:2008:FCC:1376616.1376746}, dbPedia \cite{auer2007dbpedia}, YAGO \cite{yago3}, WikiData \cite{vrandevcic2014wikidata}, aggregate information about millions of entities into a graph of [subject, predicate, object] RDF triples.
The problem with KBs is that they are inherently incomplete and miss a lot of entities, facts and predicates.
In addition, triple data representation format complicates retrieval of KB concepts relevant to question phrases.
The focus of the proposed research in factoid question answering lies on the idea of combining structured KB and unstructured text data, which can help a QA system to overcome these drawbacks.


% THIS PIECE IS GOOD, BUT IT DUPLICATES SOMETHING I HAVE ALREADY SAID. KEEPING IT JUST IN CASE...
% Billions of documents on the web contain all kinds of knowledge about the world, which can be retrieved to answer user questions.
%However, each individual statement includes a very limited amount of information about mentioned entities.
%On the other side, modern open domain large scale knowledge bases, such as dbPedia\footnote{http://wiki.dbpedia.org/}, YAGO\cite{yago3}, Freebase\footnote{http://www.freebase.com}, WikiData\footnote{https://www.wikidata.org/}, etc., contain millions of entities and facts about them, and are quite effective in answering some of the user questions.
%However, knowledge bases have their own disadvantages:
%\begin{itemize}
%\item knowledge bases are inherently incomplete \cite{Dong:2014:KVW:2623330.2623623}, even the largest existing resources miss a lot of entities, facts and properties, that might be of interest to some users.
%\item it's quite challenging to translate a natural language question into a structured language, such as SPARQL, to query a knowledge base \cite{BerantCFL13:sempre}.
%\end{itemize}

The main challenge in non-factoid question answering lies in the diversity of question and answer types.
One of the most effective strategies is to reuse answers to previously asked questions, which could be found, for example, in CQA archives \cite{Shtok:2012:LPA:2187836.2187939}.
Unfortunately, it's not always possible to find a similar question, that has already been answered, because many information needs are unique in general or in details.
Alternative strategies include ranking text passages extracted from retrieved web documents.
One of the main challenges of this approach is estimating semantic similarity between the question and an answer candidate \cite{soricut2006automatic}.
Therefore, one would benefit from knowing what kind of questions could a paragraph of text answer.
This information can often be inferred from the structure of a web page, e.g. forums, FAQ pages, or estimated using title, subtitle and other page elements.
Therefore, one of the questions I'm going to focus in my thesis is how to effectively use the structure of web page to predict whether an extracted passage of text answer the given question.

However, ranking isn't the only important part of the question answering pipeline.
A system can only rank and return a good answer if it was able to retrieve relevant information from a collection.
Non-factoid questions, especially those that people post on CQA websites are often long, which makes it problematic to use directly as search queries.
Previous research has studied certain question transformation strategies \cite{AgichteinLG01,brill_askmsr,lin2003question}, however the focus was on shorter factoid questions.
In my thesis I would like to focus on the problem of query generation for non-factoid questions using some recent advances in deep learning.
Another promising direction of research, which I'm going to explore in my thesis, is answer generation, \ie by summarizing the information a system could retrieve.
Different answer candidates might by complimentary to each other, answer different parts of the question or provide complimentary opinions on the subject.

Unfortunately, no matter how good a QA system is, there will always be cases, when it is impossible to return a satisfactory answer to user's question, \eg existing data sources might not contain necessary information or the question may simply be ambiguous or poorly worded.
In the former situation a QA system can appeal to an alternative external data source, \eg other people via crowdsourcing, while in the later scenario a system should probably reply to the user with some clarification question or give some kind of feedback on how she could solve her information need.


\section{Research Questions}

Research questions I proposed addressed in my thesis are the following:

\begin{enumerate}
\item RQ1. How to effectively combine unstructured text and structured knowledge base data to improve factoid question answering?
% THESE ARE SUBQUESTIONS, I DON'T THINK I NEED TO STATE THEM HERE.
% \begin{enumerate}
%  \item What types of questions can be answered using text, KB or a combination of both?
%  \item How does semantic annotation of unstructured data compare to information extraction for question answering?
%     (information extraction for KB construction vs open information extraction vs unstructured data annotation)
%  \item How does a combination of structured and unstructured data sources improve each of the main QA system components: question analysis, candidate generation, evidence extraction and answer selection?
%\end{enumerate}
\item RQ2. What kind of information about a web page can help scoring a passage extracted from it as a possible answer to the given question?
\item RQ3. How to build question and answer summarization models to improve candidate retrieval and answer generation for non-factoid question answering?
\item RQ4. How we can improve user experience with question answering systems for complex informational tasks?
\end{enumerate}


\section{Research Plan}

\subsection{Combining KB and Text Data for Factoid Question Answering (Chapter \ref{chapter:factoid})}
\label{sec:plan1}

The goal is to study a problem of using multiple structured KB and unstructured data together to improve factoid question answering.
Two major issues with KBQA is knowledge base incompleteness and complexity of translating natural language question into a structured query.
Text documents on the other hand are easier to match against the question, contain more information than a typical knowledge base, but aggregating information across multiple statements and documents is complicated.

One way to improve the situation with knowledge base incompleteness is to extract missing information from other data sources, \eg \cite{Cafarella:2008:WEP:1453856.1453916,Cafarella:2009:WES:1519103.1519112,Dong:2014:KVW:2623330.2623623,Etzioni:2008:OIE:1409360.1409378,Gupta:2014:BOS:2732286.2732288,kushmerick1997wrapper}.
I propose to explore one additional data source, that wasn't used for relation extraction before, namely question-answer pairs.
Section \ref{sec:relation_extraction} will describe our experiments and results in utilizing this data to improve knowledge base coverage.
Unfortunately, relation extraction isn't perfect either and there are both precision and recall losses.
Therefore, I propose to explore semantic annotation of entity mentions as a way to bridge the gap between KB entity graph and text documents.
Such representation will allow us to do simple string matching on text documents and at the same time explore the knowledge about the mentioned entities in KB and vice versa. Section \ref{sec:text+kb} describes the approach in more detail.

\subsection{Using Web Page Information to Improve Passage Ranking in Non-factoid Question Answering (Chapter \ref{sec:non-factoid:architecture:page-structure})}
\label{sec:plan2}

To answer RQ2 I'm planning to study what kind of information from web pages can be useful to predict whether a passage of text answer the given questions.
First, I'll derive a dataset of questions from TREC LiveQA'15 with passages, labeled by TREC assessors, and extract the corresponding web pages.
This allows us to set the problem as passage ranking problem using a set of passage and web page context features.
I will design a set of features representing a passage and some key elements from the web page and train an answer ranking model.
The feature ablation experiments will reveal the relative importance of different model components.

\subsection{Question and Answer Summarization for Non-factoid Question Answering (Chapter \ref{sec:non-factoid:architecture:analysis})}
\label{sec:plan3}

The goal is to develop models for question summarization, which should improve candidate answer retrieval performance, and answer summarization to generate the final response of the system.
The plan is to explore recent advances in deep learning for text summarization \cite{rush-chopra-weston:2015:EMNLP} and generation \cite{karpathy2015deep} and apply these techniques for the above mentioned problems.
The effectiveness of these models will be evaluated using the data from TREC LiveQA'15 and tested inside this year challenge model.


\subsection{Interaction between a QA system and humans (Chapter \ref{chapter:users})}
\label{sec:plan4}

In my thesis I'm planning to consider three different types of interactions between a QA system and humans: crowdsourcing, providing users with strategic hints to help them structure their search process, and clarification questions, which a QA system can ask users to resolve certain ambiguities.

For crowdsourcing, I'm going to study how a QA system can leverage a pool of human workers to crowdsource some data, which can help it answer certain difficult questions (Section \ref{sec:non-factoid:crowd}).
Automated QA sytems operate in near real-time, which poses certain challenges for crowdsourcing.
First, I'll study if it is possible to get useful data from crowd workers under a certain time limit, and then implement an almost real-time crowdsourcing system to help an automated system answer questions from TREC LiveQA 2016 shared task.

Strategic search hints are certain suggestions, which an automated system can provide to a user to structure her search task, formulate easier questions that a system can tackle.
I will study how a user react to different types of hints and how the hints affect the overall task success rate (Section \ref{sec:user:hints}).

Finally, for questions that an automated system is unable to understand it make sense to come back to the user with some clarification questions rather than a totally useless answer.
In Section \ref{sec:user:clarification} I study what kind of clarification questions real users ask, and how a system can generate a certain frequent subset of them automatically.


\subsection{Research Timeline}

\noindent
A tentative timeline for the work that needs to be done is shown below:

\begin{itemize}
\item Completing the work proposed in Sections \ref{sec:plan2} and \ref{sec:plan3} (4/2016 - 5/2016): develop individual components for question analysis, answer candidate retrieval and answer generation and integrate them into a system to participate in TREC LiveQA'16.
\item Completing the work proposed in Section \ref{sec:plan1} (6/2016 - 7/2016): Develop a model to use annotation of entity mentions for factoid question answering and compare it against existing techniques. I'm also planning to develop a new QA dataset, which will include more diverse and realistic set of questions than existing KBQA datasets and larger than available TREC QA datasets.
\item Completing the work proposed in Section \ref{sec:plan4} (4/2016 - 7/2016): Integrate a crowdsourcing module into my TREC LiveQA'16 system as one of the options, develop a model to predict ambiguous questions on a CQA website and propose certain clarification questions.
\item Thesis writing (08/2016 - 09/2016) 
\item Thesis defense (10/2016)
\end{itemize}


\section{Contributions and Implications}

The key contributions of the proposed research are:

% ADD HOW EACH OF THE POINTS COULD BE USEFUL FOR THE FIELD

\begin{itemize}
\item A novel model for relation extraction from archives of question-answer pairs
\item New hybrid KB-Text question answering approach, that improves knowledge base question answering by using information from unstructured text data sources, annotated with KB entity mentions, which essentially introduces a new types of edges into a knowledge graph
% that operates by searching an entity graph, built from both facts from a KB and links between KB entities and text fragments mentioning these entities
\item New dataset for entity-centric factoid question-answering built from an archive of CQA question-answer pairs
\item A non-factoid question-answering system, that incorporates novel question and answer summarization components, as well as novel candidate answer ranking features, based on the information extracted from the structure of the source web document
\item New method for answer collection and rating using crowdsourcing for a near real-time question answering system
\item a study of the effect of strategic search hints on the user experience and success rate for complex informational tasks
\item A novel model for detecting ambiguous questions and formulating clarifications
\end{itemize}

% chap2.tex
%

\mychapter{Related Work}
\label{chap:related}

The field of automatic questions answering has a long history of research and dates back to the days when the first computers appear.
By the early 60s people have already explored multiple different approaches to question answering and a number of text-based and knowledge base QA systems existed at that time \cite{Simmons:1965:AEQ:363707.363732,Simmons:1970:NLQ:361953.361963}.
In 70s and 80s the development of restricted domain knowledge bases and computational linguistics theories facilitated the development of interactive expert and text comprehension systems \cite{androutsopoulos1995natural,shortliffe1975model,woods1977lunar,wilensky1988berkeley}.
The modern era of question answering research was motivated by a series of Text Retrieval Conference (TREC\footnote{http://trec.nist.gov}) question answering shared tasks, which was organized annually since 1999 \cite{voorhees2001trec}.
A comprehensive survey of the approaches from TREC QA 2007 can be found in \cite{dang2007overview}.
An interested reader can refer to a number of surveys to track the progress made in automatic question answering over the years  \cite{hirschman2001natural,andrenucci2005automated,wang2006survey,Kolomiyets:2011:SQA:2046840.2047162,prager2006open,allam2012question,gupta2012survey}.
% There are a number of works, focusing on the future research directions in QA, \eg \cite{burger2001issues}.

The main focus of research in automatic question answering was on factoid questions.
However, recently we can observe an increased interest in non-factoid question answering, and as an indicator in 2015 TREC started a LiveQA shared task track\footnote{http://trec-liveqa.org/}, in which the participant systems had to answer various questions coming from real users of Yahoo! Answers\footnote{http://answers.yahoo.com/} in real time.

In the rest of the chapter I will describe related work in factoid (Section \ref{sec:rel_work:factoid}) and non-factoid (Section \ref{sec:rel_work:nonfactoid}) question answering with the focus on data sources used.

\section{Factoid question answering}
\label{sec:rel_work:factoid}

Since the early days of automatic question answering researches explored different sources of data, which lead to the development of two major approaches to factoid question answering: text-based (TextQA) and knowledge base question answering (KBQA) \cite{Simmons:1965:AEQ:363707.363732}.
We will first describe related work in TextQA (Section \ref{sec:rel_work:factoid:text}), then introduce KBQA (Section \ref{sec:rel_work:factoid:kb}) and in Section \ref{sec:rel_work:factoid:hybrid} present existing techniques for combining different information sources together.

\subsection{Text-based question answering}
\label{sec:rel_work:factoid:text}

A traditional approach to factoid question answering over text document collections, popularized by TREC QA task, starts by querying a collection with possibly transformed question and retrieving a set of potentially relevant documents, which are then used to identify the answer.
Information retrieval for question answering has certain differences from traditional IR methods \cite{keikha2014retrieving}, which are usually based on keyword matches.
A natural language question contains certain information, that is not expected to be present in the answer (\eg the keyword who, what, when, \etc), and the answer statement might use language that is different from the question (lexical gap problem).
On the other side, there is a certain additional information about expected answer statement, that a QA system might infer from the question (\eg we expect to see in a number in response to the ``how many'' question).
One way to deal with this problem is to transform the question in certain ways before querying a collection \cite{AgichteinLG01,brill_askmsr}.
Raw text data might be extended with certain semantic annotations by applying part of speech tagger, semantic role labeling, named entity recognizer, \etc.
By indexing these annotations a question answering system gets an opportunity to query collection with additional attributes, inferred from the question \cite{bilotti2007structured,yao2013automatic}.

The next stage in TextQA is to select sentences, that might contain the answer.
One of the mostly used benchmark datasets for the task, proposed in \cite{wang2007jeopardy}, is based on TREC QA questions and sentences retrieved by participating systems\footnote{A table with all known benchmark results and links to the corresponding papers can be found on http://aclweb.org/aclwiki/index.php?title=Question\_Answering\_(State\_of\_the\_art)}.
The early approaches for the task used simple keyword match strategies \cite{ittycheriah2001ibm,soubbotin2001patterns}.
However, in many cases keywords doesn't capture the similarity in meaning of the sentences very well and researches started looking on syntactic information.
Syntactic and dependency tree edit distances and kernels allow to measure the similarity between the structures of the sentences \cite{punyakanok2004mapping,shen2005exploring,heilman2010tree,yao2013answer,wang2010probabilistic}.
Recent improvements on the answer sentence selection task come are associated with the deep learning techniques, \eg recursive neural networks using sentence dependency tree \cite{iyyer2014neural}, convolutional neural networks \cite{yu2014deep,santos2016attentive}, recurrent neural networks \cite{tan2015lstm,WangN15}.
Another dataset, called WikiQA \cite{yang2015wikiqa}, raises a problem of answer triggering, \ie detecting cases when the retrieved set of sentences don't contain the answer.

To provide a user with the concise answer to his factoid question QA systems extract the actual answer phrase from retrieved sentences.
This problem is often formulated as a sequence labeling problem, which can be solved using structured prediction models, such as CRF \cite{yao2013answer}, or as a node labeling problem in an answer sentence parse tree \cite{malon2013answer}.

Unfortunately, passages include very limited amount of information about the candidate answer entities, \ie very often it doesn't include the information about their types (person, location, organization, or more fine-grained CEO, president, basketball player, \etc), which is very important to answer question correctly, \eg for the question ``\textit{what country will host the 2016 summer olympics?}'' we need to know that \texttt{Rio de Janeiro} is a city and \texttt{Brazil} is the country and the correct answer to the question.
Therefore, a lot of effort has been put into developing answer type typologies \cite{hovy2000question,Hovy:2002:QTS:1289189.1289206} and predicting and matching expected and candidate answer types from the available data \cite{LiRoth02,li2006learning, prager2006question}.
Many approaches exploited external data for this task, I will describe some of this efforts in Section \ref{sec:rel_work:factoid:hybrid}.

Very large text collections, such as the Web, contain many documents expressing the same information, which makes it possible to use a simpler techniques and rely on redundancy of the information.
\texttt{AskMSR} QA system was one of the first to exploit this idea, and achieved very impressive results on TREC QA 2001 shared task \cite{brill2002analysis}.
The system starts by transforming a question into search queries, extracts snippets of search results from a web search engine, and consider word n-grams as answer candidates, ranking them by frequency.
A recent revision of the AskMSR QA system \cite{tsai2015web} introduced several improvements to the original system, \ie named entity tagger for candidate extraction, and additional semantic similarity features for answer ranking.
It was also observed, that modern search engines are much better in returning the relevant documents for question queries and query generation step is no longer needed.
Another notable systems, that used the web as the source for question answering are \texttt{MULDER}\cite{kwok2001scaling}, \texttt{Aranea} \cite{lin2003question}, and a detailed analysis of what affects the performance of the redundancy-based question answering systems can be found in \cite{lin2007exploration}.

\subsection{Knowledge base question answering}
\label{sec:rel_work:factoid:kb}

Earlier in the days knowledge bases were relatively small and contained information specific to a particular domain, \eg baseball statistics \cite{green1961baseball}, lunar geology \cite{woods1977lunar}, geography \cite{zelle1996learning}.
However, one of the main challenges in KBQA is mapping between natural language phrases to the database concepts, which raises a problem of domain adaption of question answering systems.

Recent development of large scale knowledge bases (\eg dbPedia \cite{auer2007dbpedia}, Freebase \cite{Bollacker:2008:FCC:1376616.1376746}, YAGO \cite{suchanek2007yago}, WikiData\footnote{http://www.wikidata.org} shifted the attention towards open domain question answering.
Knowledge base question answering approaches can be evaluated on an annual Question Answering over Linked Data (QALD\footnote{www.sc.cit-ec.uni-bielefeld.de/qald/}) shared task, and some popular benchmark dataset, such as Free917 \cite{cai2013large} and WebQuestions \cite{BerantCFL13:sempre}.
A survey of some of the proposed approaches can be found in \cite{unger2014introduction}.

A series of QALD evaluation campaigns has started in 2011, and since then a number of different subtasks have been offered, \ie since QALD-3 includes a multilingual task, and QALD-4 formulated a problem of hybrid question answering.
These tasks usually use dbPedia knowledge base and provide a training set of questions, annotated with the ground truth SPARQL queries.
The hybrid track is of particular interest to the topic of this dissertation, as the main goal in this task is to use both structured RDF triples and free form text available in dbPedia abstracts to answer user questions.

% DIFFERENCES OF HYBRID TRACK AND THIS WORK
% Questions in the hybrid track are manually created in such a way, that they can \textit{only} be answered using a combination of RDF and free text data.
% Secondly, the hybrid task focuses on text data already present in a KB, whereas we are exploring external text resources.
% In general, because of the expensive labeling process, QALD datasets are rather small, for example, QALD-5 training set for multilingual question answering includes 300 examples and 40 examples for the hybrid task.
% The evaluation was performed on 50 questions for multilingual task and just 10 for hybrid.

The problem of lexical gap and lexicon construction for mapping natural language phrases to knowledge base concepts is one of the major difficulties in KBQA.
The earlier systems were mainly trained from question annotated with the correct parse logical form, which is expensive to obtain.
Such an approach is hard to scale to large open domain knowledge bases, which contain millions of entities and thousands of different predicates.
An idea to extend a trained parser with additional lexicon, built from the Web and other resources, has been proposed by \cite{CaiY13}.
However, most of the parses of the question produce different results, which means that it is possible to use question-answer pairs directly \cite{BerantCFL13:sempre}.
PARALEX system (\cite{fader2013paraphrase}) construct a lexicon from a collection of question paraphrases from WikiAnswers\footnote{https://answers.wikia.com/}.
A somewhat backwards approach was proposed in ParaSempre model of \cite{BerantL14:parasempre}, which ranks candidate structured queries by first constructing a canonical utterance for each query and then using a paraphrasing model to score it against the original question.
Another approach to learn term-predicate mapping is to use patterns obtained using distant supervision \cite{mintz2009distant} labeling of a large text corpus, such as ClueWeb \cite{yao2014information}.
Such labelled collections can also be used to train a KBQA system, as demonstrated by \cite{ReddyLS14}.
Such an approach is very attractive as it doesn't require any manual labeling and can be easily transfered to a new domain.
However, learning from statements instead of question answer pairs has certain disadvantages, \eg question-answer lexical gap and noise in distant supervision labeling.
Modern knowledge bases also contain certain surface forms for their predicates and entities, which makes it possible to convert KB RDF triples into questions and use them for training \cite{BordesCW14:emnlp}.
Finally, many systems work with distributed vector representations for words and RDF triples and use various deep learning techniques for answer selection.
A common strategy is to use a joint embedding of text and knowledge base concepts.
For example, character n-gram text representation as input to a convolutional neural network can capture the gist of the question and help map phrases to entities and predicates \cite{yih2014semantic}.
Joint embeddings can be trained using multi-task learning, \eg a system can learn to embed a question and candidate answer subgraph using question-answer pairs and question paraphrases at the same time (\cite{BordesCW14:emnlp}).
Memory Networks, developed by the Facebook AI Lab, can also be used to return triples stored in network memory in a response to the user question \cite{bordes2015large}.
This approach uses embeddings of predicates and can answer relatively simple questions, that do not contain any constraints and aggregations.
To extend deep learning framework to more complex questions, \cite{dong2015question} use multi-column convolutional neural network to capture the embedding of the entity path, context and type.

As for the architecture of KBQA systems, two major approaches have been identified: semantic parsing and information extraction.
Semantic parsing starts from question utterances and work to produce the corresponding semantic representation, \eg logical form.
The model of \cite{BerantCFL13:sempre} uses a CCG parser, which can produce many candidates on each level of parsing tree construction.
A common strategy is to use beam search to keep top-k options on each parsing level or agenda-based parsing \cite{berant2015imitation}, which maintains current best parses across all levels.
An alternative information extraction strategy was proposed by \cite{YaoD14}, which can be very effective for relatively simple questions.
A comparison of this approaches can be found in \cite{yao2014freebase}.
The idea of the information extraction approach is that for most of the questions the answer lies in the neighborhood of the question topic entity.
Therefore, it is possible to use a relatively small set of query patterns to generate candidate answers, which are then ranked using the information about how well involved predicates and entities match the original question.

Question entity identification and disambiguation is the key component in such systems, they cannot answer the question correctly if the question entity isn't identified.
Different systems used NER to tag question entities, which are then linked to a knowledge base using a lexicon of entity names \cite{BerantCFL13:sempre,BerantL14:parasempre,xu2014answering}.
However, NER can easily miss the right span and the whole system would fail to produce the answer.
Recently, most of the state-of-the-art system on WebQuestions dataset used a strategy to consider a reasonable subset of token n-grams, each of which can map to zero or more KB entities, which are disambiguated on the answer ranking stage \cite{yao-scratch-qa-naacl2015,bastmore:cikm:2015:aquu,yih:ACL:2015:STAGG}.
Ranking of candidates can be done using a simple linear classification model \cite{yao-scratch-qa-naacl2015} or a more complex gradient boosted trees ranking model \cite{bastmore:cikm:2015:aquu,yih:ACL:2015:STAGG}.

Some questions contain certain conditions, that require special filters or aggregations to be applied to a set of entities. 
For example, the question ``\textit{who won 2011 heisman trophy?}'' contains a date, that needs to be used to filter the set of heisman trophy winners, the question ``\textit{what high school did president bill clinton attend?}'' requires a filter on the entity type to filter high schools from the list of educational institutions, and ``\textit{what is the closest airport to naples florida?}'' requires a set of airports to be sorted by distance and the closest one to be selected.
Information extraction approaches either needs to extend the set of candidate query templates used, which is usually done manually, or to attach such aggregations later in the process, after the initial set of entities have been extracted \cite{yih:ACL:2015:STAGG}.
An alternative strategy to answer complex questions is to extend RDF triples as a unit of knowledge with additional arguments and perform question answering over n-tuples \cite{yin2015answering}.
In \cite{wang2015large} authors propose to start from single KB facts and build more complex logical formulas by combining existing ones, while scoring candidates using paraphrasing model.
Such a a template-free model combines the benefits of semantic parsing and information extraction approaches.

% PROPOSAL
% However, most of the models are still biased towards the types of questions present in the training set and would benefit from more training data.
% In this work I propose to extend the training set with question-answer pairs available on CQA websites, which were shown to be useful for relation extraction \cite{SavenkovLDA15}.
% In addition, I propose to use unlabeled text resources for candidate query ranking, which can help to generalize to unseen types of questions and questions about predicates never mentioned in the training set.

\subsection{Hybrid question answering}
\label{sec:rel_work:factoid:hybrid}

A natural idea of combining available information sources to improve question answering has been explored for a long time.
WordNet lexical database \cite{miller1995wordnet} was among the first resources, that were adapted by QA community \cite{hovy2001use,pasca2001informative}, and it was used for such tasks as query expansion and definition extraction.
Wikipedia\footnote{http://www.wikipedia.org}, which can be characterized as an unstructured and semi-structured (infoboxes) knowledge base, quickly became a valuable resource for answer extraction and verification \cite{ahn2005using,buscaldi2006mining}.
Developers of the Aranea QA system noticed that structured knowledge bases are very effective in answering a significant portion of relatively simple questions \cite{lin2003question}.
They designed a set of regular expressions for popular questions that can be efficiently answered from a knowledge base and fall back to regular text-based methods for the rest of the questions.

One of the major drawbacks of knowledge bases is their incompleteness, which means that many entities, predicates and facts are missing from knowledge bases, which limits the number of questions one can answer using them.
One approach to increase the coverage of knowledge bases is to extract information from other resources, such as raw text\cite{MintzBSJ09,jijkoun2004information,Gupta:2014:BOS:2732286.2732288}, web tables\cite{cafarella2008webtables}, \etc.
However, the larger the knowledge base gets, the more difficult it's to find a mapping from natural language phrases to KB concepts.
Alternatively, open information extraction techniques (\cite{Etzioni:2008:OIE:1409360.1409378}) can be used to extract a surface form-based knowledge base, which can be very effective for question answering.
Open question answering approach of \cite{Fader:2014:OQA:2623330.2623677} combines multiple structured (Freebase) and unstructured (OpenIE) knowledge bases together by converting them to string-based triples.
User question can be first paraphrased using paraphrasing model learned from WikiAnswers data, then converted to a KB query and certain query rewrite rules can be applied, and all queries are ranked by a machine learning model.

SPOX tuples, proposed in \cite{yahya2013robust}, encode subject-predicate-object triples along with certain keywords, that could be extracted from the same place as RDF triple.
These keywords encode the context of the triple and can be used to match against keywords in the question. The method attempts to parse the question and uses certain relaxations (removing SPARQL triple statements) along with adding question keyphrases as additional triple arguments.
As an extreme case of relaxation authors build a query that return all entities of certain type and use all other question terms to filter and rank the returned list.

However, by applying information extraction to raw text we inevitably lose certain portion of the information due to recall errors, and extracted data is also sometimes erroneous due to precision errors.
In \cite{xu2016enhancing}, authors propose to use textual evidence to do answer filtering in a knowledge base question answering system.
On the first stage with produce a list of answers using traditional information extraction techniques, and then each answer is scored using its Wikipedia page on how well it matches the question. 
Knowledge bases can also be incorporated inside TextQA systems.
Modern KBs contain comprehensive entity types hierarchies, which were utilized in QuASE system of \cite{Sun:2015:ODQ:2736277.2741651} for answer typing.
In addition, QuASE exploited the textual descriptions of entities stored in Freebase knowledge base as answer supportive evidence for candidate scoring.
However, most of the information in a KB is stored as relations between entities, therefore there is a big potential in using all available KB data to improve question answering.

Another great example of a hybrid question answering system is IBM Watson, which is arguably the most important and well-known QA systems ever developed so far.
It was designed to play the Jeopardy TV show\footnote{https://en.wikipedia.org/wiki/Jeopardy!}.
The system combined multiple different approaches, including text-based, relation extraction and knowledge base modules, each of which generated candidate answers, which are then pooled together for ranking and answer selection.
The full architecture of the system is well described in \cite{ferrucci2010building} or in the full special issue of the IBM Journal of Research and Development \cite{ibm_watson_special_issue}.
YodaQA \cite{baudivs2015yodaqa} is an open source implementation of the ideas behind the IBM Watson system.

% The main difference between such systems and the proposed research is that hybrid systems typically use separate pipelines to extract candidates from different sources and only merge the candidate set while ranking.
% I propose to extend the representation of each of the data sources for better candidate generation from the beginning.

% [!!!QALD HYBRID TRACK]

\section{Non-factoid question answering}
\label{sec:rel_work:nonfactoid}

ANSWER REUSING
Some questions on CQA websites are repeated very often and answers can easily be reused, \cite{Liu:2008:USA:1599081.1599144} studies different types of CQA questions and answers and analyzes them with respect to answer re-usability.
A number of methods for similar question retrieval have been proposed [NAMELY???].
Alignment between question terms can serve as a good indicator of their semantic similarity.
Such an alignment can be produced using a machine learning model with a set of features, representing the quality of the match \cite{wang2015faq}.
Alignment and translation models are usually based on term-term similarities, which are often computed from a monolingual alignment corpus.
This data can be very sparse, and to overcome this issue \cite{fried2015higher} proposed higher-order lexical semantic models, which estimates similarity between terms by considering paths of length more than 1 on term-term similarity graph.
Monolingual alignment corpora are also limited, however, it's possible to use the discourse relations of sentences in a text to learn monolingual alignment models \cite{sharp2015spinning}.

Questions often have some metadata, such as category on a community question answering website.
This information can be very useful for certain disambiguations, and can be encoded in the answer ranking model \cite{zhou2015learning}.

WEB SEARCH BASED RETRIEVAL
One of the first non-factoid question answering system was described in \cite{soricut2006automatic} and was based on web search using chunks extracted from the original question.
The ranking of extracted answer candidates was done using a translation model, which showed better results than n-gram based match score.


RANKING
Candidate answer passages ranking problem becomes even more difficult in non-factoid questions answering as systems have to deal with larger piece of text and need to ``understand'' what kind of information is expressed there.
One of the first extensive studies of different features for non-factoid answer ranking can be found in \cite{surdeanu2011learning}, who explored information retrieval scores, translation models, tree kernel and other features using tokens and semantic annotations (dependency tree, semantic role labelling, \etc) of text paragraphs.

WebAP is a dataset for non-factoid answer sentence retrieval, which was developed in \cite{yang2016beyond}.
Experiments conducted in this work demonstrated, that classical retrieval methods doesn't work well for this task, and multiple additional semantic (ESA, entity links) and context (adjacent text) features have been proposed to improve the retrieval quality.

The structure of the web page, from which the answers are extracted can be very useful.
Wikipedia articles have a good structure, and the information encoded there can be extracted in a text-based knowledge base, which can be used for question answering \cite{sondhi2014mining}.

TREC LIVEQA

Most of the approaches from TREC LiveQA 2015 combined similar question retrieval and web search techniques \cite{ecnucs_liveqa15,savenkov_liveqa15,diwant_liveqa15}.
Answers to similar questions are very effective for answering new questions \cite{savenkov_liveqa15}.
However, we a CQA archive doesn't have any similar questions, we have to fall back to regular web search.
The idea behind the winning system of CMU \cite{diwant_liveqa15} is to represent each answer with a pair of phrases - clue and answer text.
Clue is a phrase that should be similar to the given question, and the passage that follows should be the answer to this question.

\section{User interactions}

An interesting approach for knowledge base construction through dialog with the user has been proposed by \cite{hixon2015learning}.

A very nice crowdsourcing method to obtain answers to tail information needs was proposed by \cite{bernstein2012direct}.
Question query-url pairs are first mined from query logs, and then the wisdom of a crowd is used to extract and save answers to these questions.

\cite{braunstain2016supporting} retrieves Wikipedia statements that support user answers for opinion questions.

\section{Summary of Related Work}

Most previous work in ...

\clearpage
% chap3.tex
%

\newif\ifcompress
\compresstrue   % Uncomment this line for the authors
\compressfalse % Uncomment these two lines for anonymous review

\mychapter{Joint reasoning over text and KB for answer generation}

%Previous work has largely ignored a key problem in question recommendation, i.e., whether the potential answerer is likely to accept and answer the recommended questions in a timely manner. 

\noindent 
In this work I propose to enrich the input data representation for QA systems by combining available unstructured, semi-structured and structured data sources for joint reasoning, which can improve the performance of question answering over both text collections and knowledge bases.


\section{Joint reasoning over text and KB for answer generation}


\subsection{Text-based QA}

\begin{figure}
\centering
\includegraphics[width=1.0\textwidth]{figures/text_kb}
\caption{Annotation of natural language text with mentioned entities and their subgraphs in a knowledge base}
\label{fig:text_kb}
\end{figure}

For question answering over text corpora I propose to extend the text representation with annotations about mentioned entities and their relations from open \cite{Fader:2014:OQA:2623330.2623677} or schema-based knowledge bases (e.g. dbPedia or Freebase).
Such representation allows not only find different mentions of the same entity, but also look into the connections of the mentioned entities in order to learn more about the candidate answer.
For example, for the question mentioned in the introduction \textit{``What republican senators supported the nomination of Harriet Miers to the Supreme Court?''} and a candidate answer sentence \textit{``Minority Leader Harry Reid had already offered his open support for Miers.''}, such joint text-KB representation can look like Figure \ref{fig:text_kb}.
A QA system can discover that ``Harry Reid'' political affiliation is with the Democratic Party, and he cannot be referred to as ``republican senator''.
In other cases using a KB as an additional source of information may reveal specific connections between entities in the question and in the answer candidates.
For example, for another TREC QA 2007 question \textit{``For which newspaper does Krugman write?''} and retrieved candidate answer \textit{New York Times} a path between ``Paul Krugman'' and ``New York Times'' in the knowledge graph gives an evidence in support of the candidate.

More specifically, to do this kind of inference I propose:
\begin{itemize}
\item use existing approaches for document retrieval (e.g. web search using question as a query \cite{tsai2015web}) and candidate answer extraction.
\item perform entity linking to mentions of KB entities in questions and corresponding candidate answers.
\item for each mentioned entity extract a subgraph containing its neighborhood up to certain number of edges away and paths to other mentioned entities.
\item follow machine learning approach for candidate answer ranking and extend the feature representation with features derived from subgraph analysis. Examples of features are:
	\begin{itemize}
	\item features describing discovered connections between entities mentioned in a question and a candidate answer, such as indicators of the relations, combination of relations with words and n-grams from the questions, similarity between the relations and the question text (using tf-idf or embeddings representation), etc. Textual representations of the predicates in structured knowledge bases can be obtained either from its description or using patterns learned from a large collection using distant supervision \cite{MintzBSJ09}.
	\item features describing the entities mentioned in the answer, i.e. similarities between entity properties and question words, n-grams and phrases, etc.
	\end{itemize}
\end{itemize}

For training text-based QA model I propose to use available QnA pairs from community question answering websites, which represent real user tasks and after certain filtering can be a good fit for training both factoid and non-factoid question answering systems.
The data can help to learn more associations between the language used in questions and their corresponding answers, which can be encoded as conditional probabilities (e.g. $p(w_a|w_{q_1},...,w_{q_n}$, where $w_a$ is a word of the answer and $w_{q_i}$ is some subset of the question words), pointwise mutual information or by employing deep learning techniques \cite{WangN15}.

\subsection{Knowledge base QA}

\begin{figure}
\centering
\includegraphics[width=1.0\textwidth]{figures/kb_text}
\caption{Annotation of KB graph nodes and edges with unstructured text data}
\label{fig:kb_text}
\end{figure}

\begin{table}
\centering
\caption{Example of a question from WebQuestions dataset with related unstructured information}
\begin{tabular}{| p{5cm} | p{11cm} |} \hline
Question & Who is the woman that John Edwards had an affair with?\\
\hline
Provided answer & ``Writer'', ``Politician'', ``Lawyer'', ``Attorneys in the United States''\\
\hline
Correct answer & Rielle Hunter\\
\hline
Phrase from Wikipedia & John Edwards had engaged in an affair with Rielle Hunter...\\
\hline
QnA pair from Yahoo! Answers & Who was it that John Edwards had an affair with? Today, John Edwards admitted to having an affair with filmmaker Rielle Hunter.\\
\hline
\end{tabular}
\label{table:kbqa_example}
\end{table}

Lexicons learned during training of a knowledge base question answering systems are limited and often needs to be retrained to include additional data.
To complement the lexicon learned during training for candidate structured query scoring I propose to use unstructured text data related to mentioned entities and predicates (see Figure \ref{fig:kb_text}).
For example, Table \ref{table:kbqa_example} shows an example of a question from the WebQuestions dataset, that is answered incorrectly by a state-of-the-art system.
A similar question is missing from the training set, however, an easy web search can retrieve a relevant sentence, which give enough supporting evidence to answer this question correctly.

More specifically, I propose:
\begin{itemize}
\item Use one of the available state of the art systems, such as \cite{bastmore:cikm:2015:aquu}, as a baseline.
\item Extend a set of features representing a candidate answer with features derived from unstructured text fragments:
	\begin{itemize}
	\item from large document collection (such as the web) retrieve a set of passages by querying a search system with question, question + answer, question and answer entities as queries.
	\item find mentions of answer entities in the passages and use some aggregated statistics as features for the corresponding candidates.
	\item for each candidate answer retrieve a set of patterns used to express the corresponding predicates obtained using distant supervision from a large text collection and compute the similarities between these patterns and the question text.
	\end{itemize}
\end{itemize}

In addition, similar to text-based systems, I propose to include QnA pairs from CQA websites as weakly labeled training data.
For example, a question similar to the one presented in Table \ref{table:kbqa_example} is not present in the labeled training set, but can easily be found in Yahoo! Answers\footnote{http://answers.yahoo.com/}.
CQA data should be preprocessed (possibly filtered by categories of the questions and other heuristics), and select QnA pairs mentioning at least one entity in the question and in the answer.
After a reasonable cleanup such QnA pairs can be used for training treating answer entities as the correct answer.

% Finally, to solve the problem of compositionality of queries following the idea proposed in \cite{ReddyLS14} learn lexical features from raw sentences and their distantly supervised alignments to a KB, but avoid expensive and innaccurate semantic parsing step and learn direct associations between surface features and KB elements.
% \textbf{Actually I don't have a very good idea how to do this, probably need to remove this paragraph}.

\section{Summary}


% chap4.tex
%

\mychapter{Non-factoid Question Answering}

Most of the questions that people have are not factoid and cannot be simply answered with a name, date or a number.
Typically, such questions require a more elaborate fragment of text as an answer.
Traditionally, question answering systems turn to web documents that might contain some relevant passages to be used as an answer.

\noindent
In this chapter, I summarize the proposed work for improving automatic non-factoid question answering by better understanding the structure of web documents and relationships between their parts and fragments.
% In particular, we first focus on ...., and then try to ...
% The goal is 

\section{Utilizing the Structure of Web Pages}

Non-factoid questions are typically answered with a relatively long paragraph of text\footnote{TREC LiveQA'15 challenge limits the answer to 1000 characters}.
This fact and the nature of questions limits the utility of structured KB resources.
One of the main challenges for non-factoid question answering is matching between the question needs and the information expressed in text fragment.
Analysis of TREC LiveQA 2015 participants \cite{savenkov_liveqa15} revealed that the quality of answers extracted from previously posted similar questions is typically higher than from regular web passages.
Therefore, non-factoid QA system would benefit from the information on which questions does a paragraph of text answer.
This information can often be extracted from the structure of a web document, e.g. forum threads, FAQ pages or various CQA websites.
Alternatively, we can train a model to predict whether a paragraph answers a given question using titles, subtitles and surrounding text of a web page.

\begin{figure}[h]
\centering
\includegraphics[width=\textwidth]{img/web_page_structure_nonfactoid}
\label{fig:web_page_structure_nonfactoid}
\caption{Using web page structure information for non-factoid question answering}
\end{figure}

My proposal for non-factoid question answering can be summarized as follows:
\begin{itemize}
\setlength\itemsep{0em}
\item \textbf{CQA candidate generation}: retrieve a set of question-answer pairs by searching a CQA archive\footnote{https://answers.yahoo.com/}
\item \textbf{Web document retrieval}: retrieve a set of documents by querying web search with the question (and queries generated from it)
\item \textbf{Web candidate answer generation}: classify web page into one of the following types: article, forum thread, FAQ page, CQA page, other. Extract key elements using type-specific extractors (QnA pairs, FAQ and CQA pages, forum question and posts and article passages with the corresponding titles, subtitles and surrounding text).
\item \textbf{Ranking}: Rank the generated candidate answers by building on techniques from existing research \cite{surdeanu2011learning}.
\end{itemize}


\section{Evaluation}
LiveQA


\section{Summary}

% chap5.tex
%

\mychapter{Human Interaction with Question Answering Systems}
\label{chapter:users}

\noindent

Modern automatic question answering systems are still far from AI machines, that we often imagine or see in the movies.
Many user information needs are still unanswered by existing techniques.
For example, only 36\% of answers of a winning approach from TREC LiveQA 2015 shared task were judged good or excellent.
And it's unlike to become 100\% as users are not perfect either and often the questions they ask are hard to understand or ambiguous.
Therefore, it is important to improve not only the answering aspect of the systems, but also interaction experience altogether.
In this chapter I propose a couple of research directions in this area, which can help improve the overall success rate by engaging in a dialog with the user.
More specifically, in Section \ref{sec:user:hints} I describe strategic search hints for complex informational tasks, which a system can show to the user in case its response was not satisfactory.
We discuss the effects such hints have on the user success rate and satisfaction.
However, this type of intervention leaves all the heavy lifting of formulating good search queries on the user.
Section \ref{sec:user:clarification} discuss how a system can engage in a dialog by asking some clarification questions to resolve ambiguities in user's question.

%-=-=-=-=-=-=-=-=-=-=-=-=-=-=-=-=-=-=-=-=-=-=-=-=-=-=-=-=-=-=-=-=-=-=-=-
\section{Search Hints for Complex Informational Tasks}
\label{sec:user:hints}

Search engines are ubiquitous, and millions of people of varying experience use them on daily basis.
Unfortunately, not all searches are successful.
Bilal and Kirby \cite{Bilal:2002:DSI:637512.637516} reported that about half of the participants of their user study felt frustration when searching.
Xie and Cool \cite{xie2009understanding} demonstrated that most of the time users have problems with formulating and refining search queries.
Besides good retrieval performance, a successful search requires users to possess certain skills.
Search skills can be trained, e.g. Google offers a course\footnote{http://www.powersearchingwithgoogle.com} on improving search efficiency.
Although very useful, such courses are time consuming and detached from real search problems of these particular users.
Displaying search hints is another technique that has both learning effect, and offers immediate assistance to the user in solving her current search task.
Moraveji et al. \cite{Moraveji:2011:MIU:2009916.2009966} demonstrated that hints, suggesting certain search engine functionality, help people find answers more quickly, and the effect is retained after a week without hints.

% Besides the awareness about search tools available, adopting general search strategies is extremely important when dealing with a difficult search task.
In my thesis I propose to explore {\em strategic} search hints, that are designed to guide a user in solving her search problem.
More specifically, we chose the divide-and-conquer strategy, \ie splitting an original difficult question into smaller problems, searching answers to the subtasks and combining them together.
Two sets of strategic hints were manually designed: {\em generic} hints describing the divide-and-conquer strategy in general and {\em task-specific} hints providing a concrete strategy to solve the current search task.
To evaluate the effect of the hints on behavior and search success we conducted a user study with 90 participants.
The results of the user study demonstrate that well-designed task-specific hints can improve search success rate.
In contrast, generic search hints, which were too general and harder to follow, had negative effect on user performance and satisfaction.

\subsection{User Study}

\begin{figure}
\centering
\includegraphics[width=0.75\textwidth]{img/ufindit}
\caption{The interface of the search game used in the study}
\label{figure:ufindit}
\end{figure}

To estimate the effect of strategic search hints on user behavior we conducted a study in a form of a web search game similar to ``a Google a Day''\footnote{http://www.agoogleaday.com/} and uFindIt \cite{Ageev:2011:FYG:2009916.2009965}. Participants were hired using Amazon Mechanical Turk\footnote{http://www.mturk.com/}. 

The goal of the web search game used in the user study is to find answers to several questions with the provided web search interface (Figure \ref{figure:ufindit}). 
Players are instructed not to use any external tools.
% Figure \ref{figure:ufindit} shows the interface of the game.
The questions are given one by one and since tasks might be too difficult, a chance to skip a question was provided, although users were instructed that effort put into solving a question will be evaluated.
To answer a question each player needs to provide a link to a page containing the answer as well as its text.
The answer is automatically verified and a popup box notifies a player if the answer is incorrect (since the answer can be formulated differently, presence of a keyword was checked).
A player can then continue searching or skip the question when she gives up.
A bonus payment was made to players who answer all questions correctly.
We used Bing Search API\footnote{http://www.bing.com/toolbox/bingsearchapi} as a back-end of the game search interface.
All search results and clicked documents were cached so users asking the same query or clicking the same page got the same results.
At the end of the game a questionnaire was presented asking for feedback on user satisfaction with the game, prior experience and other comments.

\begin{table}[tbh]
\centering
\caption{Search tasks used for the study, and specific search hints shown to one of the user groups}
\label{table:tasks}
\begin{tabular}{|p{1cm}|p{4.5cm}|p{4.2cm}|p{6.0cm}|} \hline
 & Question & Correct Answer & Specific hints \\ \hline
Task 1 & I can grow body back in about two days if cut in half. Many scientists think I don't undergo senescence. What am I? & Senescence means ``biological aging''. Hydra is considered biologically immortal and regenerates fast. & \parbox[t]{6cm}{
1. Find what is senescence\\
2. Find who does not undergo senescence\\
3. Find who can also regenerate body and choose the one that satisfies both conditions} \\ \hline
Task 2 & Of the Romans "group of three" gods in the Archaic Triad, which one did not have a Greek counterpart? & Archaic Triad includes Jupiter, Mars and Quirinus. Among those Quirinus didn't have a Greek counterpart. &
\parbox[t]{6cm}{
1. Find the names of the gods from the Archaic triad\\
2. For each of the gods find a Greek counterpart
}\\ \hline
Task 3 & As George surveyed the ``waterless place'', he unearthed some very important eggs of what animal? & "Gobi" in Mongolian means ``Waterless place''. The first whole dinosaur eggs were discovered there in 1923. & \parbox[t]{6cm}{
1. Find what is the ``waterless place'' mentioned in the question?\\
2. Search for important eggs discovery in this ``waterless place''}\\ \hline
Task 4 & If you were in the basin of the Somme River at summers end in 1918, what language would you have had to speak to understand coded British communications? & Cherokee served as code talkers in the Second Battle of the Somme. & \parbox[t]{6cm}{
1. Find the name of the battle mentioned in the questions\\
2. Search for which coded communications language was used in this battle\\
} \\ \hline
\end{tabular}
\end{table}

The tasks for the study were borrowed from the ``A Google a Day'' questions archive.
Such questions are factual, not ambiguous and usually hard to find the answer with a single query, which makes them interesting for user assistance research.
We filtered search results to exclude all pages that discuss solutions to ``A Google a Day'' puzzles.
To do this we removed pages that mention a major part of the search question or ``a google a day'' phrase.
To keep users focused throughout the whole game we limited the number of questions to 4.
The tasks are described in Table \ref{table:tasks} and were presented to all participants in the same order to ensure comparable learning effects.

The questions have multiple parts and to solve them it is helpful to search for answers to parts of the questions and then combine them.
In one of the previous studies we observed, that most of the users didn't adopt the divide-and-conquer strategy, but kept trying to find the ``right'' query.
We decided to estimate the effect of strategic search hints, suggesting users to adopt the new strategy.

We built 2 sets of strategic hints: \textit{task specific} and \textit{generic}.
Task-specific hints described steps of one of the possible solutions to each question (Table \ref{table:tasks}).
Second set contained a single hint, which was shown for all tasks. Generic hint described the divide-and-conquer strategy:\\
\hrule
\begin{enumerate} \itemsep0pt \parskip0pt \parsep0pt
\item Split the question into 2 or more logical parts
\item Find answers to the parts of the question
\item Use answers to the parts of the question to find answer to the full question
\end{enumerate}

For example, the question: ``The second wife of King Henry VIII is said to haunt the grounds where she was executed. What does she supposedly have tucked under her arm?''
\begin{enumerate} \itemsep0pt \parskip0pt \parsep0pt
\item Search [second wife King Henry VIII] to find Anne Boleyn.
\item Search [Anne Boleyn under arm] to find that her ghost is in the London Tower where she is said to carry her head tucked underneath her arm.
\end{enumerate}
\hrule

To control for the learning effect demonstrated in \cite{Moraveji:2011:MIU:2009916.2009966}, each user was assigned to one of the three groups:
\begin{enumerate}\itemsep0pt \parskip0pt \parsep0pt
\item users who didn't get any hints
\item users who got task-specific hints
\item users who got the generic hints
\end{enumerate}


\subsection{Results}

From 199 unique participants, who clicked the HIT on Amazon Mechanical Turk only 90 players finished the game.
We further examined all games manually and filtered out 9 submissions for one of the following reasons: lack of effort (e.g. skipped several tasks after none or a single query) or usage of external resources (e.g. the answer was obtained without submitting any queries or results explored didn't contain the answer).
Furthermore, 10 players from the group which received hints indicated in the survey that they didn't see them, so we filtered out those submissions and finally we had 71 completed games (29 for no hints, 20 for task-specific hints and 22 for generic hints groups).

\subsubsection{Effects of Search Tips on Performance}

In order to measure search success rate we looked at the number of questions answered correctly by different groups of users\footnote{Since users were allowed to skip a question we are counting the number of questions that were eventually solved correctly even if a player made some incorrect attempts}.
Figure \ref{figure:hints:task_success} shows that success rate is higher for users who saw task-specific hints compared to users who didn't get such assistance.
Surprisingly, having the generic hint decreased the success rate, although users could easily ignore a hint they didn't like.
A possible explanation is: generic hints were harder to follow and users who tried and failed became frustrated and didn't restart their searches.

% Similar to \cite{Moraveji:2011:MIU:2009916.2009966} we looked at the average time to answer a question (for this analysis we removed games where a user didn't find the answer and skipped the task).

The plot of average time to answer a question on Figure \ref{figure:hints:task_time} doesn't show an improvement for the task-specific hints group, except for the question 1.
Our task-specific hints represent a possible way to solve a problem and there is no guarantee, that it is the fastest one.
It is worth noting, that users from the generic search hint group had slightly higher variance in success time, which can probably be explained by the fact that some users were successful in finding the right way to follow the hint and some other users struggled with it much longer.
Another insight comes from the number of incorrect attempts users made.
Figure \ref{figure:hints:incorrect} demonstrates the average number of incorrect answer attempts for all groups of users.
Although the variance is high, there is a tendency for users who saw task-specific hints to make less attempts than both other groups.
This is not in direct correspondence with time spent on the game.
It seems that the users who saw a clear strategy to solve the question were less likely to notice plausible, but incorrect solution.
Moreover, we analyzed texts of incorrect answers, and can conclude that a big part of incorrect submission are due to users trying all possible options they found on the way, even if these options are clearly wrong.
% We should note, that unfortunately we didn't limit the number of attempts per problem, thus strategy to verify an answer by submitting it made sense.

\begin{figure}[h]
\centering
  \begin{subfigure}{0.32\textwidth}
  \includegraphics[width=\textwidth]{img/success_per_task}
  \caption{Success rate per task for each group of participants}
  \label{figure:hints:task_success}
  \end{subfigure}
  \begin{subfigure}{0.32\textwidth}
  \centering
  \includegraphics[width=\textwidth]{img/time_per_task}
  \caption{Task completion time for each group of players}
  \label{figure:hints:task_time}
  \end{subfigure}
  \begin{subfigure}{0.32\textwidth}
  \centering
  \includegraphics[width=\textwidth]{img/incorrect}
  \caption{The number of incorrect submission attempts per question for all groups of users}
  \label{figure:hints:incorrect}
  \end{subfigure}
\caption{Results of the user study on the effectiveness of strategic search tips on search task success rate}
\label{fig:hints:results}
\end{figure}

\begin{figure}[h]
\centering
\begin{subfigure}[t]{0.32\textwidth}
	\includegraphics[scale=0.26]{img/liked}
	\caption{How did you like the game?}
    \label{figure:survey:liked}
\end{subfigure}
\begin{subfigure}[t]{0.32\textwidth}
	\includegraphics[scale=0.26]{img/difficult}
	\caption{How difficult was the game?}
    \label{figure:survey:difficult}
\end{subfigure}
\begin{subfigure}[t]{0.32\textwidth}
	\includegraphics[scale=0.26]{img/useful}
	\caption{Were search hints useful to you?}
    \label{figure:survey:useful}
\end{subfigure}
\caption{Proportions of replies to some of the survey question for each group of users}
\label{figure:hints:survey}
\end{figure}

We also looked at other search behavior characteristics: number of queries submitted, number of clicks made, average length of the queries. The variance in these characteristics was too high to make any speculations regarding their meaning.

\subsubsection{Effects of Search Tips on User Experience}

Finally, we looked at the surveys filled out by each group of users.
Figure \ref{figure:hints:survey} presents proportions of different answers to three of the questions: ``How did you like the game?'', ``How difficult was the game?'' and ``Were search hints useful to you?''.
Surprisingly, user satisfaction with the game was lower for users who saw hints during the game and users who didn't get any assistance enjoyed it more.
The replies to the question about game difficulty are in agreement with the success rate: users who saw task-specific hints rated difficulty lower than participants who struggled to find the correct answers.
The game was very difficult on average, however, some participants from the group who received task-specific hints surprisingly rated it as very easy, which suggests that our hints do help users.
This is supported by the answers to the last question on whether hints were helpful (Figure \ref{figure:survey:useful}).

To summarize, the results of the conducted user study suggest that specific search hints can be helpful, which is indicated by higher success rate, lower number of incorrect attempts and positive feedback in the end of study survey.
In contrast, generic hints can have negative effect on user experience, which is indicated by lower success rate, increased number of incorrect attempts and higher perceived tasks complexity according to the survey.

\subsection{Summary}
In this section we studied the effect of strategic search hints on user behavior. 
The conducted user study in a form of a web search game demonstrated the potential of good hints in improving search success rate.
However, to be useful, they should be designed carefully.
Search hints that are too general can be detrimental to search success.
We also find that even searchers who are more effective using specific search hints, feel subjectively less satisfied and engaged than the control group, indicating that search assistance has to be specific and timely if it is to improve the searcher experience.

We should note, that specific search hints used in this work were manually generated and an interesting question of future work is how to generate such useful hints automatically.
It should be possible to learn strategies applied by the experienced search users and suggest them to the rest.

%-=-=-=-=-=-=-=-=-=-=-=-=-=-=-=-=-=-=-=-=-=-=-=-=-=-=-=-=-=-=-=-=-=-=-=-
\section{Clarification Questions}
\label{sec:user:clarification}

Nowadays, with intelligent assistants and chat bots we are observing a shift towards natural language interfaces, which will enable richer interaction between a human and computer, in particular question answering.
Most of the existing systems are one-sided, \ie they operate by returning an answer in a response to a user question.
A richer model will inevitably lead to a dialogue rather than request-response kind of communication.
There are many practical aspects of maintaining a dialogue for question-answering.
For example, many questions that user ask are not clear or ambiguous, \eg ``How can I bring up my pictures that I had in windows 8.1?''.
Instead of returning a useless answer, a system can detect a problem with the question and come back with a clarification question to the user, which would allow to use the response to generate a better answer.

The research I propose towards a dialog-based QA are:
\begin{itemize}
\item Study user behavior patterns associated with asking clarification questions
\item Build a classification model to predict when a question requires a clarification
\item Choose a subset of clarification types and design a system to generate questions in response to ambiguous user queries.
\end{itemize}

For our behavior study we are planning to use data from StackExchange CQA website\footnote{http://stackexchange.com/}, which allows users not only answer the posted questions, but also comment on them and many comments are indeed clarification questions (Figure \ref{figure:user:clarification:stackexchange}).
We will filter out questions, that contain a question comment and perform an exploratory study of different types of forms of these responses.

Preliminary analysis suggested, that there are many different kinds and nature of clarifications.
Some address the general quality of user's question, \eg ``Is this a duplicate of [...]''.
Another fraction of the clarifications address certain ambiguities present in the question or in the described situation, \eg ``In what way is it oversize -- too high, wide, or both? What are the dimensions?''.
These types of clarifications doesn't usually refer to any particular answer or solution.
Finally, some comments are actually proposing certain answers, which might be trivial (``Have you tried contacting the company's customer support?'') or might not work (``Have you tried adjusting the fill valve / float arm so the tank fills higher?'').

\begin{figure}[h]
\centering
\includegraphics[width=0.7\textwidth]{img/stackexchange_example}
\caption{A question and clarification comment posted by users on StackExchange question answering website}
\label{figure:user:clarification:stackexchange}
\end{figure}


In my thesis I'm planning to focus on a subset of questions, that are ambiguous and therefore cannot be reliably answered without a clarification.
We will select a subset of such questions from the dataset, and train a machine learning model to predict whether a question requires a clarification or can be answered as is.
This problem is somewhat similar to the question answerability, studied in \cite{dror2013will,shah2010evaluating}.
However, the problem here is more specific, as we can imagine that an ambiguous question can still receive an answer, and questions that are not ambiguous can be left unanswered, for example if the community did not know the answer.
However, the features explored in these works will probably be useful in our scenario as well.
For evaluation, I will split the original dataset into training and test sets and use precision and recall classification metrics.

Finally, the next stage after we detected an ambiguous question is to automatically ask a clarification question.
I'm planning to explore two possible approaches: template-based and recurrent neural networks based approaches.
Template-based approach will target a subset of potential questions, \eg ``What type of <OBJ> do you have?'', ``How old is your <OBJ>?'', \etc.
Such templates can be automatically mined from the collection, and the only problem for the model is to detect the object to ask about, which can be solved by training another machine learning model.

Recurrent neural networks showed impressive results in multiple areas, such as machine translation \cite{sutskever2014sequence}, image caption generation \cite{vinyals2015show} and dialogs \cite{vinyals2015neural}.
The later work is especially relevant and for this experiment I'm planning to build on it use a bidirectional LSTM model with soft-attention to generate a clarification given an ambiguous question.

Evaluation of this part of the work is more complicated, because someone needs to judge the usefulness of the generated clarification questions.
We are planning to use crowdsourcing to label each question - clarification question pair.
We will ask workers to first decide if the model chose the target of the clarification question correctly, \ie if the target of the clarification question indeed makes the question ambiguous.
Then workers will judge if the text of the clarification question is reasonable and the answer to it can resolve the ambiguity.

%-=-=-=-=-=-=-=-=-=-=-=-=-=-=-=-=-=-=-=-=-=-=-=-=-=-=-=-=-=-=-=-=-=-=-=-
\section{Summary}

This chapter described some results and proposed work aimed at improving user experience with the question answering system.
Strategic hints can help the user to split a complex informational task into smaller pieces, which an automated system can handle.
The results of our experiments suggest that hints, that specifically address user's current search task can indeed lead to the overall task success, however generic hints might be detrimental to user experience.
Another way of engaging in a question-answering dialog is to ask clarification questions when the question is ambiguous.
The proposed research can serve as a good starting point for understanding how people use clarifications in question answering and how a system can generate them automatically.
% chap6.tex
%

\mychapter{Summary and Discussion}
\label{chapter:discussion}

\noindent

In my thesis I propose several pieces of work towards improving user satisfaction with question answering systems.
I plan to consider factoid and non-factoid questions, as usually classified in the community, separately, because certain techniques and data sources are most useful for one type of questions and not another.

The research I'm propose to conduct for improving factoid question answering targets a problem of combining information available in different data sources, \ie structured knowledge bases and unstructured text documents.
Semantic annotations of entity mentions in documents create additional connections between knowledge base entities, which should improve KB coverage and allow a system to answer more questions and with better precision.
However, such an approach have certain potential limitations.
\begin{itemize}
\item A set of additional links for some entities is likely to be big, which makes it impossible to explore them all. Information extraction approaches on the other hand aggregate information over the whole collection, although the extraction process itself brings extra noise.
\item ....
\end{itemize}

For non-factoid question answering I propose several improvements, targeting different stages of the question answering process.
Question to query generation neural network model, trained to improve the document retrieval performance, should increase the recall of the question answering system by identifying the key phrases that needed to be search for.
The answer passage scoring module should achieve a better precision by analyzing the structure of the answer origin web page, and detecting question-answer pairs and other key structural elements.
Finally, the proposed direction in automatic answer summarization is a way to increase the quality of answers by combining evidence from multiple different data sources, possibly providing additional information and alternative opinions.
Possible limitations of the proposed directions and approaches are:
\begin{itemize}
\item question to query generation model can be retrieval engine specific, which may force the model to be retrained after certain changes in the retrieval algorithm. An alternative strategy is to integrate a similar question summarization module into the retrieval engine itself.
\item web page structure?...
\item summarization?... The problem is that it might not work well...
\end{itemize}

Finally, I touch a user aspect of question answering, in particular user assistance with hints in case a system failed to respond to user information needs, clarification questions, which is one of the first steps in dialog-based question answering and finally using the wisdom of a crowd to improve the performance of question answering systems.
\begin{itemize}
\item Hints 
\item Clarifications
\item Crowdsourcing
\end{itemize}

%\appendix
%\include{appendixa}
%\include{appendixb}
%\include{appendixc}
%%%%%%%%%%%%%%%%%%%%%%%% STOP EDITING HERE %%%%%%%%%%%%%%%%%%%%%%%%

%\include{mybib}


\bibliographystyle{abbrv}\small 
\setlength{\bibsep}{2pt}
\singlespacing
\bibliography{References}
%\bibliography{References,searchAsk,sigproc,questionIntent,reranking}
\end{document}
