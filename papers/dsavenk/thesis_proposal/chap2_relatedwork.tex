% chap2_relatedwork.tex
%

\mychapter{Related Work}
\label{chapter:related}

\noindent

The field of automatic questions answering has a long history of research and dates back to the days when the first computers appear.
By the early 60s people have already explored multiple different approaches to question answering and a number of text-based and knowledge base QA systems existed at that time \cite{Simmons:1965:AEQ:363707.363732,Simmons:1970:NLQ:361953.361963}.
In 70s and 80s the development of restricted domain knowledge bases and computational linguistics theories facilitated the development of interactive expert and text comprehension systems \cite{androutsopoulos1995natural,shortliffe1975model,woods1977lunar,wilensky1988berkeley}.
The modern era of question answering research was motivated by a series of Text Retrieval Conference (TREC\footnote{http://trec.nist.gov}) question answering shared tasks, which was organized annually since 1999 \cite{voorhees2001trec}.
A comprehensive survey of the approaches from TREC QA 2007 can be found in \cite{dang2007overview}.
An interested reader can refer to a number of surveys to track the progress made in automatic question answering over the years  \cite{hirschman2001natural,andrenucci2005automated,wang2006survey,Kolomiyets:2011:SQA:2046840.2047162,prager2006open,allam2012question,gupta2012survey}.
% There are a number of works, focusing on the future research directions in QA, \eg \cite{burger2001issues}.

The main focus of research in automatic question answering was on factoid questions.
However, recently we can observe an increased interest in non-factoid question answering, and as an indicator in 2015 TREC started a LiveQA shared task track\footnote{http://trec-liveqa.org/}, in which the participant systems had to answer various questions coming from real users of Yahoo! Answers\footnote{http://answers.yahoo.com/} in real time.

In the rest of the chapter I will describe related work in factoid (Section~\ref{section:relatedwork:factoid}) and non-factoid (Section~\ref{section:relatedwork:non-factoid}) question answering with the focus on data sources used.
In Section~\ref{section:relatedework:crowdsourcing} I will describe existing research in crowdsourcing for question answering and crowdsourcing for real-time systems.
Finally, Section~\ref{section:relatedwork:user} explains some of the prior works on user interactions with question answering and search systems and some user assistance techniques.

\section{Factoid question answering}
\label{section:relatedwork:factoid}

Since the early days of automatic question answering researches explored different sources of data, which lead to the development of two major approaches to factoid question answering: text-based (TextQA) and knowledge base question answering (KBQA) \cite{Simmons:1965:AEQ:363707.363732}.
We will first describe related work in TextQA (Section \ref{section:relatedwork:factoid:text}), then introduce KBQA (Section \ref{section:relatedwork:factoid:kbqa}) and in Section \ref{section:relatedwork:factoid:hybrid} present existing techniques for combining different information sources together.

\subsection{Text-based question answering}
\label{section:relatedwork:factoid:text}

A traditional approach to factoid question answering over text document collections, popularized by TREC QA task, starts by querying a collection with possibly transformed question and retrieving a set of potentially relevant documents, which are then used to identify the answer.
Information retrieval for question answering has certain differences from traditional IR methods \cite{keikha2014retrieving}, which are usually based on keyword matches.
A natural language question contains certain information, that is not expected to be present in the answer (\eg the keyword who, what, when, \etc), and the answer statement might use language that is different from the question (lexical gap problem).
On the other side, there is a certain additional information about expected answer statement, that a QA system might infer from the question (\eg we expect to see in a number in response to the ``how many'' question).
One way to deal with this problem is to transform the question in certain ways before querying a collection \cite{AgichteinLG01,brill_askmsr}.
Raw text data might be extended with certain semantic annotations by applying part of speech tagger, semantic role labeling, named entity recognizer, \etc.
By indexing these annotations a question answering system gets an opportunity to query collection with additional attributes, inferred from the question \cite{bilotti2007structured,yao2013automatic}.

The next stage in TextQA is to select sentences, that might contain the answer.
One of the mostly used benchmark datasets for the task, proposed in \cite{wang2007jeopardy}, is based on TREC QA questions and sentences retrieved by participating systems\footnote{A table with all known benchmark results and links to the corresponding papers can be found on http://aclweb.org/aclwiki/index.php?title=Question\_Answering\_(State\_of\_the\_art)}.
The early approaches for the task used simple keyword match strategies \cite{ittycheriah2001ibm,soubbotin2001patterns}.
However, in many cases keywords does not capture the similarity in meaning of the sentences very well and researches started looking on syntactic information.
Syntactic and dependency tree edit distances and kernels allow to measure the similarity between the structures of the sentences \cite{punyakanok2004mapping,shen2005exploring,heilman2010tree,yao2013answer,wang2010probabilistic}.
Recent improvements on the answer sentence selection task come are associated with the deep learning techniques, \eg recursive neural networks using sentence dependency tree \cite{iyyer2014neural}, convolutional neural networks \cite{yu2014deep,santos2016attentive}, recurrent neural networks \cite{tan2015lstm,WangN15}.
Another dataset, called WikiQA \cite{yang2015wikiqa}, raises a problem of answer triggering, \ie detecting cases when the retrieved set of sentences do not contain the answer.

To provide a user with the concise answer to her factoid question, QA systems extract the actual answer phrase from retrieved sentences.
This problem is often formulated as a sequence labeling problem, which can be solved using structured prediction models, such as CRF \cite{yao2013answer}, or as a node labeling problem in an answer sentence parse tree \cite{malon2013answer}.

Unfortunately, passages include very limited amount of information about the candidate answer entities, \ie very often it does not include the information about their types (person, location, organization, or more fine-grained CEO, president, basketball player, \etc), which is very important to answer question correctly, \eg for the question ``\textit{what country will host the 2016 summer olympics?}'' we need to know that \texttt{Rio de Janeiro} is a city and \texttt{Brazil} is the country and the correct answer to the question.
Therefore, a lot of effort has been put into developing answer type typologies \cite{hovy2000question,Hovy:2002:QTS:1289189.1289206} and predicting and matching expected and candidate answer types from the available data \cite{LiRoth02,li2006learning, prager2006question}.
Many approaches exploited external data for this task, I will describe some of these efforts in Section~\ref{section:relatedwork:factoid:hybrid}.

Very large text collections, such as the Web, contain many documents expressing the same information, which makes it possible to use a simpler techniques and rely on redundancy of the information.
\texttt{AskMSR} QA system was one of the first to exploit this idea, and achieved very impressive results on TREC QA 2001 shared task \cite{brill2002analysis}.
The system starts by transforming a question into search queries, extracts snippets of search results from a web search engine, and consider word n-grams as answer candidates, ranking them by frequency.
A recent revision of the AskMSR QA system \cite{tsai2015web} introduced several improvements to the original system, \ie named entity tagger for candidate extraction, and additional semantic similarity features for answer ranking.
It was also observed, that modern search engines are much better in returning the relevant documents for question queries and query generation step is no longer needed.
Another notable systems, that used the web as the source for question answering are \texttt{MULDER}\cite{kwok2001scaling}, \texttt{Aranea} \cite{lin2003question}, and a detailed analysis of what affects the performance of the redundancy-based question answering systems can be found in \cite{lin2007exploration}.

\subsection{Knowledge base question answering}
\label{section:relatedwork:factoid:kbqa}

Earlier in the days knowledge bases were relatively small and contained information specific to a particular domain, \eg baseball \cite{green1961baseball}, lunar geology \cite{woods1977lunar}, geography \cite{zelle1996learning}.
However, one of the problems of techniques developed in this period is domain adaptation, as it is quite challenging to map from natural language phrases to database concepts in open domain when the search space is quite large.
Recent development of large scale knowledge bases (\eg dbPedia \cite{auer2007dbpedia}, Freebase \cite{Bollacker:2008:FCC:1376616.1376746}, YAGO \cite{suchanek2007yago}, WikiData\footnote{http://www.wikidata.org} shifted the attention towards open domain question answering.
Knowledge base question answering approaches can be evaluated on an annual Question Answering over Linked Data (QALD\footnote{www.sc.cit-ec.uni-bielefeld.de/qald/}) shared task, and some popular benchmark dataset, such as Free917 \cite{cai2013large} and WebQuestions \cite{BerantCFL13:sempre}.
A survey of some of the proposed approaches can be found in \cite{unger2014introduction}.

A series of QALD evaluation campaigns has started in 2011, and since then a number of different subtasks have been offered, \ie since 2013 QALD includes a multilingual task, and QALD-4 formulated a problem of hybrid question answering.
These tasks usually use dbPedia knowledge base and provide a training set of questions, annotated with the ground truth SPARQL queries.
The hybrid track is of particular interest to the topic of this dissertation, as the main goal in this task is to use both structured RDF triples and free form text available in dbPedia abstracts to answer user questions.

The problem of lexical gap and lexicon construction for mapping natural language phrases to knowledge base concepts is one of the major difficulties in KBQA.
The earlier systems were mainly trained from question annotated with the ground truth logical forms, which are expensive to obtain.
Such approaches are hard to scale to large open domain knowledge bases, which contain millions of entities and thousands of different predicates.
An idea to extend a trained parser with additional lexicon, built from the Web and other resources, has been proposed by Q. Cai and A. Yates~\cite{CaiY13}.
However, most of the parses of a question produce different results, which means that it is possible to use question-answer pairs directly~\cite{BerantCFL13:sempre}.
PARALEX system of A.Fader \etal~\cite{fader2013paraphrase} constructs a lexicon from a collection of question paraphrases from WikiAnswers\footnote{https://answers.wikia.com/}.
A somewhat backward approach was proposed in ParaSempre model of J.Berant \etal~\cite{BerantL14:parasempre}, which ranks candidate structured queries by first constructing a canonical utterance for each query and then using a paraphrasing model to score it against the original question.
Another approach to learn term-predicate mapping is to use patterns obtained using distant supervision~\cite{MintzBSJ09} labeling of a large text corpus, such as ClueWeb~\cite{yao2014freebase}.
Such labeled collections can also be used to train a KBQA system, as demonstrated by S.Reddy \etal~\cite{ReddyLS14}.
This approach is very attractive as it does not require any manual labeling and can be easily transfered to a new domain.
However, learning from statements instead of question answer pairs has certain disadvantages, \eg question-answer lexical gap and noise in distant supervision labeling.
Modern knowledge bases also contain certain name or surface forms for their predicates and entities, which makes it possible to convert KB RDF triples into questions and use them for training~\cite{BordesCW14:emnlp}.
Finally, many systems work with distributed vector representations for words and RDF triples and use various deep learning techniques for answer selection.
A common strategy is to use an embedding of text and knowledge base concepts into the same space.
For example, character n-gram text representation as input to a convolutional neural network can capture the gist of the question and help map phrases to entities and predicates~\cite{yih2014semantic}.
Joint embeddings can be trained using multi-task learning, \eg a system can learn to embed a question and candidate answer subgraph using question-answer pairs and question paraphrases at the same time~\cite{BordesCW14:emnlp}.
Memory Networks, developed by the Facebook AI Lab, can also be used to return triples stored in network memory in a response to the user question~\cite{bordes2015large}.
This approach uses embeddings of predicates and can answer relatively simple questions, that do not contain any constraints and aggregations.
A nice extension of this idea is so called key-value memory networks~\cite{miller2016key}, which simplify retrieval by replacing a single memory cell, which has to be selected using softmax layer, with a key-value pair.
Thus, one can encode subject and predicate of a KB triple as the key and let the model return the object as the value of a memory cell.
To extend deep learning framework to more complex questions, Li Dong \etal~\cite{dong2015question} used multi-column convolutional neural network to capture the embedding of entity path, context and type at the same time.
Another idea that allows memory networks to answer complex questions is multiple iterations over the memory, which allows a model to focus on different parts of the question and extend the current set of candidate facts, as shown by S.Jain ~\cite{jain2016question}.

As for the architecture of KBQA systems, two major approaches have been identified: semantic parsing and information extraction.
Semantic parsing starts from question utterances and work to produce the corresponding semantic representation, \eg logical form.
The model of J.Berant \etal~\cite{BerantCFL13:sempre} uses a CCG parser, which can produce many candidates on each level of parsing tree construction.
A common strategy is to use beam search to keep top-k options on each parsing level or agenda-based parsing~\cite{berant2015imitation}, which maintains current best parses across all levels.
An alternative information extraction strategy was proposed by~\cite{YaoD14}, which can be very effective for relatively simple questions.
The idea of the information extraction approach is that for most of the questions the answer lies in the neighborhood of the question topic entity.
Therefore, it is possible to use a relatively small set of query patterns to generate candidate answers, which are then ranked using the information about how well involved predicates and entities match the original question.
A comparison of this approaches can be found in~\cite{yao2014freebase}.

Question entity identification and disambiguation is the key component in such systems, they cannot answer the question correctly if the question entity is not identified.
Different systems used NER to tag question entities, which are then linked to a knowledge base using a lexicon of entity names~\cite{BerantCFL13:sempre,BerantL14:parasempre,xu2014answering}.
However, NER can easily miss the right span, which would not allow this question to be answered correctly.
Most of the recently developed KBQA systems used a strategy to consider a reasonable subset of token n-grams, each of which can map to zero or more KB entities.
Top entities according to some entity linking scores are kept and disambiguated only at the answer ranking stage~\cite{yao-scratch-qa-naacl2015,bastmore:cikm:2015:aquu,yih:ACL:2015:STAGG}.
Ranking of candidates can be done using a simple linear classification model~\cite{yao-scratch-qa-naacl2015} or a more complex gradient boosted trees ranking model~\cite{bastmore:cikm:2015:aquu,yih:ACL:2015:STAGG}.

Some questions contain certain conditions, that require special filters or aggregations to be applied to a set of entities. 
For example, the question ``\textit{who won 2011 heisman trophy?}'' contains a date, that needs to be used to filter the set of heisman trophy winners, the question ``\textit{what high school did president bill clinton attend?}'' requires a filter on the entity type to filter high schools from the list of educational institutions, and ``\textit{what is the closest airport to naples florida?}'' requires a set of airports to be sorted by distance and the closest one to be selected.
Information extraction approaches either need to extend the set of candidate query templates used, which is usually done manually, or to attach such aggregations later in the process, after the initial set of entities have been extracted~\cite{yih:ACL:2015:STAGG,xu2016enhancing}.
An alternative strategy to answer complex questions is to extend RDF triples as a unit of knowledge with additional arguments and perform question answering over n-tuples~\cite{yin2015answering}.
Z.Wang at al.~\cite{wang2015large} proposed to start from single KB facts and build more complex logical formulas by combining existing ones, while scoring candidates using paraphrasing model.
Such a template-free model combines the benefits of semantic parsing and information extraction approaches.

\subsection{Hybrid question answering}
\label{section:relatedwork:factoid:hybrid}

A natural idea of combining available information sources to improve question answering has been explored for a long time.
Researchers have used various additional resources, such as Wordnet~\cite{miller1995wordnet}, Wikipedia\footnote{http://www.wikipedia.org} and structured knowledge bases along with textual document collections.
WordNet lexical database was among the first resources, that were adapted by QA community for such tasks as query expansion and definition extractions~\cite{hovy2001use,pasca2001informative}.
Next, Wikipedia, which can be characterized as an unstructured and semi-structured (infoboxes) knowledge base, quickly became a valuable resource for answer extraction and verification~\cite{ahn2005using,buscaldi2006mining}.
Developers of the Aranea QA~\cite{lin2003question} system noticed that structured knowledge bases are very effective in answering a significant portion of relatively simple questions.
They designed a set of regular expressions for popular questions that can be efficiently answered from a knowledge base and fall back to regular text-based methods for the rest of the questions.

% Relation extraction from text.
One of the major drawbacks of knowledge bases is their incompleteness, which means that many entities, predicates and facts are missing from knowledge bases, which limits the number of questions one can answer using them.
One approach to increase the coverage of knowledge bases is to extract information from other resources, such as raw text~\cite{MintzBSJ09,jijkoun2004information,Gupta:2014:BOS:2732286.2732288}, web tables \cite{Cafarella:2008:WEP:1453856.1453916}, or infer from existing knowledge~\cite{lao2012reading,gardner2015efficient,bordes2011learning}.
As most of the information in the world is present in unstructured format, relation extraction from natural language text has been an active area of research for many years, and a number of supervised \cite{snow2004learning}, semi-supervised \cite{Agichtein:2000:SER:336597.336644} and unsupervised \cite{Fader:2011:IRO:2145432.2145596} methods have been proposed.
These techniques analyze individual sentences and can extract facts stated in them using syntactic patterns, sentence similarity, \etc.
In my thesis I extend existing techniques by adapting them to work on one additional type of text data, \ie QnA pairs.

% Joint representation of text and KB concepts. Universal schemas, etc.
Another relevant angle of relation extraction research is a joint representation of text and knowledge base data.
Introduction of text-based edges, extracted from sentences mentioning a pair of entities, to the Path Ranking Algorithm was demonstrated to be superior to KB data alone for knowledge base construction \cite{lao2012reading}.
Such a graph, consisting of KB entities, predicates and textual data can be viewed as heterogeneous information network, and such representation was effectively used to represent text documents for clustering and classification~\cite{wang2016text,wang2015incorporating}.
The idea of universal schemas for relation extraction is represent KB and natural language predicates with embeddings in low dimensional space.
The original work of S.Riedel \etal~\cite{riedel2013relation} by factorizing a matrix, in which rows correspond to entity pairs and columns to KB predicates and natural language phrases connecting these entity mentions in text.
This techniques were further improved by learning embeddings of individual entities~\cite{verga2016row}, which allows the model to generalize to unseen entity pairs, and compositionality-aware embeddings of natural language~\cite{toutanova2015representing} to better capture variability of the language.
Wang et al~\cite{wang2014knowledge} shows how to embed entities and words into the same space by preserving entity relations and word coocurrences in text.
These approaches aims at computing a similarity between KB predicates and the ways they are expressed in sentences, and they do not attempt to solve a problem of detecting relations not present in KB, which users might ask about, nor they are trying to cross the sentence boundary and extract information scattered across multiple sentences.
However, embedding of various modalities, such as knowledge base predicates and text, into the same space have been effectively used for different tasks, including question answering with so called memory networks~\cite{bordes2015large,miller2016key}.
The work I propose to do in my thesis for factoid question has similar objective and setup as key-value memory networks.
However, I am going to focus on open domain question answering and consider both embedding and IR-based methods to address facts in extended knowledge source.

% This is OpenIE and OpenQA.
However, the larger the knowledge base gets, the more difficult it is to find a mapping from natural language phrases to KB concepts.
Alternatively, open information extraction techniques~\cite{Etzioni:2008:OIE:1409360.1409378} can be used to extract a schema-less knowledge base, which can be very effective for question answering.
Open question answering approach of A.Fader \etal~\cite{Fader:2014:OQA:2623330.2623677,yin2015answering} combines multiple structured (Freebase) and unstructured (OpenIE) knowledge bases together by converting them to string-based triples.
User question can be first paraphrased using paraphrasing model learned from WikiAnswers data, then converted to a KB query and certain query rewrite rules can be applied, and all queries are ranked by a machine learning model.

% Extended knowledge graphs.
After the information is encoded into RDF triples in a knowledge base, we need to be able to map it back to natural language in order to answer user questions.
An idea of extended knowledge graphs ~\cite{elbassuoni2009language,yahya2013robust} is to extend the RDF triples with keywords, which could be extracted from the context of the triple in text, \eg from relation extraction model.
These keywords encode the context of the triple and can be used to match against keywords in the question.
To query such knowledge graphs authors proposed an extension of SPARQL language, which allows to specify keywords for some triple patterns.
However, such queries now require special answer ranking mechanism, \eg based on a language model idea~\cite{elbassuoni2009language}.
When answering natural language questions, its often hard to decide whether to map a phrases to some KB concepts, and which one to use.
Therefore, many translated queries might become overspecific and return no results at all because of the incorrect translation or lack of knowledge in a KB.
M.Yahya \etal~\cite{yahya2013robust,yahya2016relationship} proposed to use query relaxation techniques to reduce a set of triple patterns in translated SPARQL queries and use some of the question phrases as keywords in the query instead.
As an extreme case of such relaxation we can get a query with a single triple pattern, that retrieves all entities of certain type and them ranks them using all keywords from the question.

% Text + KB without information extraction: textual based filtering and QuASE.
However, by applying information extraction to raw text we inevitably lose certain portion of the information due to recall errors, and extracted data is also sometimes erroneous due to precision errors.
K. Xu \etal~\cite{xu2016enhancing} proposed to use textual evidence to do answer filtering in a knowledge base question answering system.
On the first stage with produce a list of answers using traditional information extraction techniques, and then each answer is scored using its Wikipedia page on how well it matches the question. 
Knowledge bases can also be incorporated inside TextQA systems.
Modern KBs contain comprehensive entity types hierarchies, which were utilized in QuASE system of~\cite{Sun:2015:ODQ:2736277.2741651} for answer typing.
In addition, QuASE exploited the textual descriptions of entities stored in Freebase knowledge base as answer supportive evidence for candidate scoring.
However, most of the information in a KB is stored as relations between entities, therefore there is a big potential in using all available KB data to improve question answering.

% QALD hybrid
QALD evaluation campaigns include a hybrid track in a couple of most recent challenges.
The goal of this track is to answer questions, that were designed in such a way, that can only be answered by a combination of knowledge base and textual data.
The targeted textual data is usually descriptions of each entity, stored in dbPedia.
These descriptions often represent an overview of the most important information about the entity and can be matched against some parts of the question.
The questions designed for this task typically contain multiple parts, one or more of which require textual resources.
An example question is: ``\textit{Who was vice president under the president who approved the use of atomic weapons against Japan during World War II?}''.
Due to this specifics and relatively small size of the dataset (QALD-5 training set for multilingual question answering includes 300 examples and 40 examples for the hybrid task) most of the systems are based on certain rules, \eg splitting the question into parts and issuing individual queries into full-text index or KB~\cite{park2015isoft,usbeck2015hawk}.
In my thesis I am focusing on more open settings, where the text does not have to come from inside the knowledge base.
In addition, real user questions tends to be more different than hand-crafted ones, which along with larger datasets allows to use machine learning-based modules for answer ranking and selection.

% Watson
Another great example of a hybrid question answering system is IBM Watson, which is arguably the most important and well-known QA systems ever developed so far.
It was designed to play the Jeopardy TV show\footnote{https://en.wikipedia.org/wiki/Jeopardy!}.
The system combined multiple different approaches, including text-based, relation extraction and knowledge base modules, each of which generated candidate answers, which are then pooled together for ranking and answer selection.
The full architecture of the system is well described in \cite{ferrucci2010building} or in the full special issue of the IBM Journal of Research and Development~\cite{ibm_watson_special_issue}.
YodaQA~\cite{baudivs2015yodaqa} is an open source implementation of the ideas behind the IBM Watson system.


\section{Non-factoid question answering}
\label{section:relatedwork:non-factoid}

During earlier days of research non-factoid questions received relatively little attention.
TREC QA tasks started to incorporate certain categories of non-factoid questions, such as definition questions, during the last 4 years of the challenge.
One of the first non-factoid question answering system was described by R. Soricut, Radu and E. Brill~\cite{soricut2006automatic} and was based on web search using chunks extracted from the original question.
The ranking of extracted answer candidates was done using a translation model, which showed better results than n-gram based match score.

The growth of the popularity of community question answering (CQA) websites, such as Yahoo! Answers, Answers.com, \etc, contributed to an increased interest of the community to non-factoid questions.
Some questions on CQA websites are repeated very often and answers can easily be reused to answer new questions, Y.Liu \etal~\cite{Liu:2008:USA:1599081.1599144} studied different types of CQA questions and answers and analyzes them with respect to answer re-usability.
A number of methods for similar question retrieval have been proposed~\cite{bernhard2009combining,Shtok:2012:LPA:2187836.2187939,duan2008searching,Jeon:2005:FSQ:1099554.1099572}.

Candidate answer passages ranking problem becomes even more difficult in non-factoid questions answering as systems have to deal with larger piece of text and need to ``understand'' what kind of information is expressed there.
WebAP is a dataset for non-factoid answer sentence retrieval, which was developed in \cite{yang2016beyond}.
Experiments conducted in this work demonstrated, that classical retrieval methods does not work well for this task, and multiple additional semantic (ESA, entity links) and context (adjacent text) features have been proposed to improve the retrieval quality.
One of the first extensive studies of different features for non-factoid answer ranking can be found in M.Surdeanu \etal~\cite{surdeanu2011learning}, who explored information retrieval scores, translation models, tree kernel and other features using tokens and semantic annotations (dependency tree, semantic role labelling, \etc) of text paragraphs.
Alignment between question and answer terms can serve as a good indicator of their semantic similarity.
Such an alignment can be produced using a machine learning model with a set of features, representing the quality of the match \cite{wang2015faq}.
Alignment and translation models are usually based on term-term similarities, which are often computed from a monolingual alignment corpus.
This data can be very sparse, and to overcome this issue~\cite{fried2015higher} proposed higher-order lexical semantic models, which estimates similarity between terms by considering paths of length more than 1 on term-term similarity graph.
An alternative strategy to overcome the sparseness of monolingual alignment corpora is to use the discourse relations of sentences in a text to learn term association models~\cite{sharp2015spinning}.

Questions often have some metadata, such as categories on a community question answering website.
This information can be very useful for certain disambiguations, and can be encoded in the answer ranking model~\cite{zhou2015learning}.
The structure of the web page, from which the answers are extracted can be very useful as well.
Wikipedia articles have a good structure, and the information encoded there can be extracted in a text-based knowledge base, which can be used for question answering~\cite{sondhi2014mining}.
Information extraction methods can also be useful for the more general case of non-factoid question answering.
For example, there is a huge number of online forums, FAQ-pages and social media, that contain question-answer pairs, which can be extracted to build a collection to query when a new question arrives~\cite{cong2008finding,Jijkoun:2005:RAF:1099554.1099571,Yang:2009:ISK:1526709.1526735,ding2008using,li2011question}.

TREC LiveQA shared task organized by Yahoo started a series of evaluation campaigns for non-factoid question answering.
The task is to develop a live question answering system to answer real user questions, that are posted to Yahoo! Answers community question answering website.
Most of the approaches from TREC LiveQA 2015 combined similar question retrieval and web search techniques~\cite{ecnucs_liveqa15,savenkov_liveqa15,diwang_liveqa15}.
Answers to similar questions are very effective for answering new questions \cite{savenkov_liveqa15}.
However, we a CQA archive does not have any similar questions, we have to fall back to regular web search.
The idea behind the winning system of CMU \cite{diwang_liveqa15} is to represent each answer with a pair of phrases - clue and answer text.
Clue is a phrase that should be similar to the given question, and the passage that follows should be the answer to this question.

Typically QA system simply rank passages and return the top scoring one as the answer.
However, in many cases such passages might either contain redundant information or no individual passage covers all the aspects of the question.
In such cases we can apply answer summarization techniques to build the final response.
Previous research focused on summarization of answers provided by the community~\cite{liu2008understanding,tomasoni2010metadata,pande2013summarizing,chan2012community,zhaochun_sparsecoding_2016}.
Y.Liu et al~\cite{liu2008understanding} investigated the idea that different types of questions might require different summarization strategies.
Some posts on CQA websites are quite long and actually contain multiple subquestions, by identifying those it is possible to group answers according to which particular subquestion do they answer and use this information for summarization~\cite{chan2012community,pande2013summarizing}.
Additionally, answers in CQA websites have some metadata, including the author of the answer, and this information can be effectively used to improve summarization as shown in~\cite{tomasoni2010metadata}.
An alternative to summarizing answers is to rank them by acknowledging diversity and novelty of aspects, covered by different answers~\cite{omari2016novelty}.
The key difference between the existing approaches work I propose to do in my thesis is the source of information to summarize.
Since I am planning to build an answer summarization module for a real QA system, it will have to deal with more diverse set of candidates, many of which will be totally irrelevant to the question, which adds additional challenges.
The work I am proposing to do is in sync with the answer distillation idea, described in the research proposal of~\cite{mitra2016distillation}.


\section{Crowdsourcing for Question Answering}
\label{section:relatedework:crowdsourcing}

Using the wisdom of a crowd to help users satisfy their information needs has been studied before in the literature.
\cite{bernstein2012direct} explored the use of crowdsourcing for offline preparation of answers to tail search queries.
Log mining techniques were used to identify potential question-answer fragment pairs, which were then processed by the crowd to generate the final answer.
This offline procedure allows a search engine to increase the coverage of direct answers to user questions.
In contrast, the focus of my thesis is on online question answering, which requires fast responses to users, who are unlikely to wait more than a minute.
Another related work is targeting a different domain, namely SQL queries.
The CrowdDB system~\cite{franklin2011crowddb} is an SQL-like processing system for queries, that cannot be answered by machines only.
In CrowdDB human input is used to collect missing data, perform computationally difficult functions or matching against the query.
In \cite{aydin2014crowdsourcing} authors explored efficient ways to combine human input for multiple choice questions from the ``Who wants to be a millionaire?'' TV show.
In this scenario going with the majority for complex questions is not effective, and certain answerer confidence weighting schemas can improve the results.  
CrowdSearcher platform of \cite{Bozzon:2012:ASQ:2187836.2187971} proposes to use crowds as a data source in the search process, which connects a searcher with the information available through the users of multiple different social platforms.

Many works have used crowdsourcing to get a valuable information that could guide an automated system for some complex tasks.
For example, entity resolution system of \cite{Whang:2013:QSC:2536336.2536337} asks questions to crowd workers to improve the results accuracy.
Using crowdsourcing for relevance judgments has been studied extensively in the information retrieval community, \eg, \cite{Alonso:2008:CRE:1480506.1480508,alonso2011design,grady2010crowdsourcing} to name a few.
The focus in these works is on document relevance, and the quality of crowdsourced judgments.
Whereas in my thesis I am investigating the ability of a crowd to quickly assess the quality of the answers in a nearly real-time setting.
The use of crowdsourcing in IR is not limited to relevance judgements.
The work of \cite{harris2013comparing} explores crowdsourcing for query formulation task, which could also be used inside an IR-based question answering system.
\cite{lease2013crowdsourcing} provides a good overview of different applications of crowdsourcing in information retrieval.

Crowdsourcing is usually associated with offline data collection, which requires significant amount of time.
Its application to (near) real-time scenarios poses certain additional challenges.
\cite{bernstein2011crowds} introduced the retainer model for recruiting synchronous crowds for interactive real-time tasks and showed their effectiveness on the best single image and creative generation tasks.
VizWiz mobile application of \cite{bigham2010vizwiz} uses a similar strategy to quickly answer visual questions.
This work builds on these ideas and uses the proposed retainer model to integrate a crowd into a real-time question answering system.
The work of \cite{Lasecki:2013:CCC:2501988.2502057,huang2015guardian} showed how multiple workers can sit behind a conversational agent named Chorus, where human input is used to propose and vote on responses.
The work I propose to do in my thesis uses similar ideas in application to non-factoid question answering, which requires more comprehensive responses from the workers.
Another use of a crowd for maintaining a dialog is presented in \cite{Bessho:2012:DSU:2392800.2392841}, who let the crowd handle difficult cases, when a system was not able to automatically retrieve a good response from the database of twitter data.
This idea is similar to the proposed research on selective crowdsourcing for QA, however, the main challenge in my work is to estimate the quality of the automatically generated answer.

\section{User Interactions with Question Answering Systems}
\label{section:relatedwork:user}

There has been considerable amount of work on user assistance for general web search and improving user experience with feedback, suggestions and hints.
Results of the study in \cite{xie2009understanding} demonstrate that in 59.5\% of the cases users need help to refine their searches or to construct search statements.
Individual term~\cite{ruthven2003survey} or query suggestion~\cite{Bhatia:2011:QSA:2009916.2010023, Cao:2008:CQS:1401890.1401995,Jones:2006:GQS:1135777.1135835} are among the most popular techniques for helping users to augment their queries.
The study in Diane Kelly et al~\cite{Kelly:2009:CQT:1571941.1572006} demonstrated that users prefer query suggestions over term relevance feedback, and that good manually designed suggestions improve retrieval performance.
Query suggestion methods usually use search logs to extract queries that are similar to the query of interest and work better for popular information needs~\cite{Bhatia:2011:QSA:2009916.2010023}.

When query or term suggestions are not available, it is still possible to help users by providing potentially useful search hints.
An adaptive tool providing tactical suggestions was presented in~\cite{Kriewel2010} and users reported overall satisfaction with its automatic non-intrusive advices.
Modern search engines have many features that are not typically used by an average user, but can be very useful in particular situations as shown in~\cite{Moraveji:2011:MIU:2009916.2009966}. The study demonstrated the potential effectiveness and teaching effect of hints.
Different from \cite{Moraveji:2011:MIU:2009916.2009966} in this thesis work I focus on a different type of hints.
Rather than suggesting to use certain advanced search tools, I explore the effectiveness of \textit{strategic} search hints, designed to suggest a strategy a user can adapt to solve a difficult information question.

Early QA studies considered users the sole proactive part asking refining questions and clarifying on system's response \cite{deboni2005}.
QA with a more active system's role was investigated within complex interactive QA (ciQA) TREC track: assessors provided additional information in various forms to live QA systems as a follow-up to initial inquiry; systems produced updated answers upon interactive sessions \cite{trec2007}.
The track outcomes were mixed: interactive phase degraded initial results in most cases; evaluation design was found not quite appropriate for interactive QA.

Kotov and Zhai~\cite{kotov2010} introduced a concept of \textit{question-guided search}, which can be seen as a variant of query suggestion scenario: in response to initial query the user is presented with a list of natural language questions that reflect possible aspects of the information need behind the query.
Tang \etal~\cite{tang2011} proposed a method for refinement questions generation consisting of two steps: 1)~refinement terms are extracted from a set of similar questions retrieved from a question archive; 2)~terms are  clustered using a WordNet-like thesaurus, cluster type (such as \textit{location} or \textit{food}) defines the question template to be used.
Sajjad \etal~\cite{sajjad12} described a framework for search over a collection of items with textual descriptions exemplified with xbox avatar assets (appearance features, clothes, and other belongings).
Multiple textual descriptions for each item were gathered via crowdsourcing; attribute--value pairs were extracted subsequently.
In online phase intermediate search results are analyzed and yes/no questions about attributes and values are generated sequentially in order to bisect the result set and finally come to the sought item.
Gangadharaiah and Narayanaswamy~\cite{gangadharaiah2013} elaborated a similar approach to search results refinement through clarification questions.
The authors considered customer support scenario using forum data.
In offline phase noun phrases, attribute--value pairs, and action tuples are extracted from forum collection.
In online phase answers to automatically generated questions help reduce the set of candidate anwsers.

% Relevance feedback for question answering
Existing information retrieval tools are not perfect and in many cases fail to return useful information.
User interactions data and implicit feedback can be very effective source of information, and allow a system to refine and come up with a better answer.
Relevance feedback for document retrieval has been on a research radar for a long time, since Rocchio~\cite{rocchio1971relevance} developed a method for adjusting the query based on available positive and negative feedback documents.
Since then a number of extensions for different retrieval models have been proposed, \eg ~\cite{salton1997improving,lavrenko2001relevance,lv2010positional,hiemstra2001relevance} to name a few.
However, relevance feedback for question answering is quite different from ad-hoc retrieval, where instead of a single response the goal is to rank documents according to their relevance.
In addition, negative feedback is more common, because if a user is satisfied with an answer, a system does not get a chance to use this information, in contrast to document retrieval, where quite often the goal is to find as many relevant documents as possible.
Negative relevance feedback has some key differences from the positive feedback, which tells us exactly what kind of information is relevant~\cite{wang2008study}.

% Some works on user satisfaction
Overall, the goal of the proposed thesis directions is to improve user satisfaction using a question answering system as a component of an intelligent assistant.
Estimating user satisfaction and studying the corresponding factors is another topic on its own, which I am leaving aside in my thesis.
An interested reader might refer to some of the existing research on the topic, \eg \cite{ong2009measurement,Liu:2008:PIS:1390334.1390417,kiseleva2016understanding}.
