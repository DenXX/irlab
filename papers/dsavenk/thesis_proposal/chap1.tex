% chap1.tex
%
% First chapter file is different from others
%
\mychapter{Introduction}
\label{chapter:intro}

\pagenumbering{arabic}
\setcounter{page}{1}
\pagestyle{myheadings}

%Uncomment to switch spacing.
%\baselineskip=5px

\newtheorem{definition}{Definition}
\newtheorem{proposition}{Proposition}


\section{Motivation}




The ability to answer user questions with precise and concise information is a hard problem with a long history of research.
Various data sources are available for candidate answer generation, two major ones are unstructured text corpora, and structured knowledge bases (e.g. dpPedia \cite{auer2007dbpedia} and Freebase \cite{Bollacker:2008:FCC:1376616.1376746}).
A hybrid approach to question answering \cite{baudivs2015modeling,ferrucci2010building} generates candidates from multiple sources, however each of them is typically processed separately and results are merged on the scoring and ranking stage when some information is already lost.
Efficient combination of different information sources has potential to improve both text and knowledge base question answering systems.
I propose to combine all the available sources together and do joint reasoning to generate better answer candidates and improve the overall question answering performance.

% Text-bases QA systems were shown to be quite effective on the TREC QA tasks as well as on other benchmark datasets \cite{dang2007overview}.
Question answering from text corpora typically starts by retrieving a set of potentially relevant documents using the question (or some transformation of the question \cite{AgichteinLG01}) as the query, and then extracting entities, phrases, sentences or paragraphs believed to be the answer to the question.
However, the information available in the retrieved pieces of text is very limited and often not enough to decide whether it can be the answer to the given question.
For example, below is one of the questions from TREC QA 2007 dataset:\\
\textit{``What republican senators supported the nomination of Harriet Miers to the Supreme Court?''}\\
A candidate answer sentence \textit{``Minority Leader Harry Reid had already offered his open support for Miers.''} mentions a senator ``Harry Reid'' and clearly says about his support of the nomination.
However, ``Harry Reid'' is not a correct answer to the question because he is a member of the Democratic party.
This information is not available in the answer candidate sentence, but it is present as one of the properties in Freebase: [Harry Reid, political\_party, Democratic party]\footnote{Actually, in Freebase the entities are connected by a path of length 2 through a mediator node. The predicates on the path are: /government/politician/party and /government/political\_party\_tenure/party}.
Therefore, by looking into the knowledge available about the mentioned entities a QA system can make a better judgment about the candidate answer.

Question answering over linked data (knowledge bases) converts a natural language question into a structured query, such as SPARQL.
The main challenge for such systems is to map words and phrases from the question to the corresponding entities and predicates from a KB.
Usually, such lexicon is built during training using ground truth question-query pairs \cite{CaiY13} or question-answer pairs \cite{BerantCFL13:sempre}.
Improvements were made by extending the lexicon using Wikipedia and patterns expressing certain predicates obtained via distant supervision \cite{bastmore:cikm:2015:aquu,BordesCW14:emnlp,ReddyLS14,yih:ACL:2015:STAGG,YaoD14}.
But still, the amount of available labeled or weakly labeled training data is much smaller than the amount of unstructured data.
This unstructured data will complement the learned lexicon, e.g. even if a question about a certain predicate wasn't seen during training, a set of text paragraphs mentioning both of the related entities can provide a QA system with enough evidence to make the correct decision.

Table \ref{table:data_procons} lists pros and cons of structured and unstructured data sources for factoid and non-factoid question answering.

\begin{table*}
\centering
\caption{Pros and cons of structured and unstructured data sources for factoid and non-factoid question answering}
\begin{tabular}{| l | p{6cm} | p{6cm} |}
\hline
 & unstructured data & structured data \\
\hline
factoid questions & \multicolumn{1}{|c|}{Text} & \multicolumn{1}{|c|}{Knowledge Bases} \\
 & + easy to match against question text & + aggregate all the information about entities\\
 & + cover a variety of different information types & allow complex queries over this data using special languages (e.g. SPARQL) \\
 & - each text phrase encodes a limited amount of information about mentioned entities & - hard to translate natural language questions into special query languages \\
&  & - KBs are incomplete (missing entities, facts and properties) \\
\hline
non-factoid questions & \multicolumn{1}{|c|}{Text} & \multicolumn{1}{|c|}{Question-Answer pairs} \\
 & + contain relevant information to a big chunk of user needs & + easy to find a relevant answer by matching the corresponding questions \\
 & - hard to extract semantic meaning of a paragraph to match against the question (lexical gap) & - cover a smaller subset of user information needs \\
\hline
\end{tabular}
\label{table:data_procons}
\end{table*}


The main focus of research in automatic question answering was on \textbf{factoid questions}, which inquire about a certain fact and can be answered with a short phrase, such as an entity name, date or number.
Such questions cover, although big and an important, but only a fraction of user information needs.
Questions that don't fall into this category are usually referred to as \textbf{non-factoid}.

----

Over the decades of research in factoid question answering, two relatively separate approaches have emerged: text-centric, or TextQA and knowledge base-centric, or KBQA.
Each approach has its own advantages and disadvantages (Table \ref{table:data_procons}).
Billions of documents on the web contain all kinds of knowledge about the world, which can be retrieved to answer user questions.
However, each individual statement includes a very limited amount of information about mentioned entities.
On the other side, modern open domain large scale knowledge bases, such as dbPedia\footnote{http://wiki.dbpedia.org/}, YAGO\cite{yago3}, Freebase\footnote{http://www.freebase.com}, WikiData\footnote{https://www.wikidata.org/}, etc., contain millions of entities and facts about them, and are quite effective in answering some of the user questions.
However, knowledge bases have their own disadvantages:
\begin{itemize}
\item knowledge bases are inherently incomplete \cite{Dong:2014:KVW:2623330.2623623}, even the largest existing resources miss a lot of entities, facts and properties, that might be of interest to some users.
\item it's quite challenging to translate a natural language question into a structured language, such as SPARQL, to query a knowledge base \cite{BerantCFL13:sempre}.
\end{itemize}

One way to improve the situation with knowledge base incompleteness is to extract missing information from other data sources, \eg \cite{Cafarella:2008:WEP:1453856.1453916,Cafarella:2009:WES:1519103.1519112,Dong:2014:KVW:2623330.2623623,Etzioni:2008:OIE:1409360.1409378,Gupta:2014:BOS:2732286.2732288,kushmerick1997wrapper}.
In my thesis I focus on one particular data source, that didn't receive enough attention in the relation extraction literature, namely question-answer pairs.
Section \ref{sec:relation_extraction} will describe our experiments and results in utilizing this data to improve knowledge base coverage.
Unfortunately, relation extraction isn't perfect either and there are both precision and recall losses.
Alternatively, in my thesis I propose a new hybrid approach to question answering, which leverages a combination of text and knowledge base data to improve every stage of question answering process (Section \ref{sec:text+kb}).

\subsection{Factoid vs Non-factoid Questions}

How to classify one versus the other...

\section{Research Questions}

\begin{enumerate}
\item What types of questions can be answered using text, KB or a combination of both?
\item How does semantic annotation of unstructured data compare to information extraction for question answering?
information extraction for KB construction vs open information extraction vs unstructured data annotation
\item How does a combination of structured and unstructured data sources improve each of the main QA system components: question analysis, candidate generation, evidence extraction and answer selection?
\item SOMETHING ABOUT NON-FACTOID
\item SOMETHING ABOUT USERS
\end{enumerate}


\section{Research Plan}

\subsection{Step 1 (Chapter 3)}
\label{sec:plan1}

\subsection{Step 2 (Chapter 4)}
\label{sec:plan2}

\subsection{Step 3 (Chapter 5)}
\label{sec:plan3}

\subsection{Research Timeline}

% \noindent A tentative timeline for the work that needs to be done is shown below:

%\begin{itemize}
%\item Completing the work proposed in Section \ref{sec:plan2} (10/2013 - 11/2013): For detecting implicit question intent in web search, build a better evaluation set, develop more features for the classifier and compare with more baselines.
%\item Completing the work proposed in Section \ref{sec:plan3} (12/2013 - 04/2014): For improving question search for queries with question intent, develop models that build better query model and learn better word-to-word translation relationships based on the query-question click data and question category information.
%\item \textbf{If time permits}, extending the work proposed in Section \ref{sec:plan3} (03/2014 - 04/2014): Extend the methods developed above for question search to achieve better query expansion and results ranking for general web search.
%\item Thesis writing (05/2014 - 08/2014) 
%\item Thesis defense (08/2014)
%\end{itemize}


\section{Contributions and Implications}

The key contributions of the proposed research are:
1. New hybrid KB-text question answering algorithm, that is based on graph search, which includes both KB links as well as text search edges to follow.
2. New labelled dataset for question answering (???)
3. New features for ranking answer candidates ???
